%\usepackage[small,sf,bf]{titlesec}
\usepackage[utf8]{inputenc}
\usepackage[british]{babel}

\usepackage[table,x11names,dvipsnames,rgb]{xcolor}
\usepackage{tikz}
\usetikzlibrary{snakes,arrows,shapes,automata}
\usetikzlibrary{positioning}

\usepackage{float}
\usepackage{graphicx}

%% use as
%%     \vertcenterimage{\includegraphics{*}}
\newcommand{\vertcenterimage}[1]{\raisebox{-.5\height}{#1}}

%% use as
%%     \flipbox{\includegraphics{*}}
\newcommand{\flipbox}[1]{\scalebox{1}[-1]{#1}}


\usepackage{hyperref}
\hypersetup{%
	pdfauthor={Zakariyya Mughal},%
	pdfpagemode={UseNone},%
	pdfpagelayout={SinglePage}%
}


\usetheme{Ilmenau}
\usecolortheme{beaver}
\usepackage{ifxetex}

\ifxetex
	% set font to Tahoma
	\usefonttheme{professionalfonts} % using non standard fonts for beamer
	\usefonttheme{serif} % default family is serif
	\usepackage{fontspec}
	\setmainfont{Tahoma}
\else
	\usepackage[T1]{fontenc}
\fi

\definecolor{CBLRed}{RGB}{205,0,0}

\colorlet{langC}{blue!40}
\colorlet{langM}{orange!20}
\colorlet{data}{Grey!10}

% Title on title slide
\setbeamerfont{title}{size = \Large}
\setbeamercolor{title}{fg = black, bg = white}

\setbeamercolor{frametitle}{fg=white,bg=CBLRed}
\setbeamercolor{section in head/foot}{fg=white,bg=CBLRed}
\setbeamercolor{subsection in head/foot}{fg=white,bg=CBLRed}
\setbeamercolor{author in head/foot}{fg=white,bg=CBLRed}
\setbeamertemplate{headline}{}
\beamertemplatenavigationsymbolsempty

\logo{}
\def\insertlogo
{%
	\color{gray}
	\begin{tabular*}{\paperwidth}{p{0.01\paperwidth}p{0.84\paperwidth}p{0.15\paperwidth}}
		&
		\parbox[c][][c]{\textwidth}{\raggedright
			\vertcenterimage{\includegraphics[width=\textwidth]{gfx/cbl-footer.png}}
		}
		&
		\parbox[c][][c]{\textwidth}{%\raggedleft
			{\large\insertframenumber} %/\inserttotalframenumber
		}
	\end{tabular*}
 }

%\usefonttheme[onlymath]{serif}
\ifx \printpresentarticle \undefined
	\setbeamertemplate{frametitle}[default][center]
	\setbeamertemplate{footline}{}
	%\setbeamertemplate{footline}[frame number]

	%% Change beamer bullets to circles rather than the ball default
	\setbeamertemplate{itemize items}[circle]
	\setbeamertemplate{enumerate items}[circle]
\fi


\usepackage{textcomp}
\usepackage{fancyvrb}
\usepackage{changepage}
\usepackage{multicol}
\usepackage{wasysym}
\usepackage{listings}

\lstset{%basicstyle=\small\ttfamily,
%numbers=left,
%escapeinside=||
}
\newenvironment{indented}{\begin{adjustwidth}{1.5em}{}}{\end{adjustwidth}}

% http://tex.stackexchange.com/questions/12550/changing-default-width-of-blocks-in-beamer/12551#12551
\newenvironment<>{varblock}[2][.9\textwidth]{%
  \setlength{\textwidth}{#1}
  \begin{actionenv}#3%
    \def\insertblocktitle{#2}%
    \par%
    \usebeamertemplate{block begin}}
  {\par%
    \usebeamertemplate{block end}%
  \end{actionenv}}

\ifx \printpresentnote \undefined
% no notes
\else
\setbeameroption{show only notes}
\fi

%% use as
%%     \af{ number of overlay }{ slide label }
%% e.g.,
%%     \af{2}{intro-slide}
\newcommand{\af}[2]{\againframe<#1|handout:0>[noframenumbering]{#2}}


%% TOC at beginning of sections/subsections
\ifx \printpresenthandout \undefined
	% no handout
	\AtBeginSubsection[]
	{
		\begin{frame}[noframenumbering]\frametitle{Table of contents}
			\begin{multicols}{2}
				\tableofcontents[
					currentsection,
					hideothersubsections,
					sectionstyle=show/shaded,
					subsectionstyle=show/shaded/hide,
					subsubsectionstyle=hide,
				]
			\end{multicols}
	  \end{frame}
	}
\else
	\AtBeginSection[]
	{
		\begin{frame}[noframenumbering]\frametitle{Table of contents}
			\begin{multicols}{2}
				\tableofcontents[
					currentsection,
					currentsubsection,
					hideothersubsections,
					sectionstyle=show/shaded,
					subsectionstyle=show/show/hide,
					subsubsectionstyle=hide,
				]
			\end{multicols}
	  \end{frame}
	}
\fi

%% TikZ arrows
%% From <https://tex.stackexchange.com/questions/61507/drawing-arrows-in-beamer>
\tikzset{
    myarrow/.style={
        draw,
        fill=orange,
        single arrow,
        minimum height=3.5ex,
        single arrow head extend=1ex
    }
}
\newcommand{\arrowup}{%
\tikz [baseline=-0.5ex]{\node [myarrow,rotate=90] {};}
}
\newcommand{\arrowdown}{%
\tikz [baseline=-1ex]{\node [myarrow,rotate=-90] {};}
}
\newcommand{\arrowright}{%
\tikz [baseline=-0.5ex]{\node [myarrow,rotate=0] {};}
}
\newcommand{\arrowleft}{%
\tikz [baseline=-0.5ex]{\node [myarrow,rotate=180] {};}
}


%% no "Figure" in \caption{}
%% simple caption
\setbeamertemplate{caption}{\raggedright\insertcaption\par}
%% with color
%\setbeamertemplate{caption}{%
%\begin{beamercolorbox}[wd=.5\paperwidth, sep=.2ex]{block
%body}\insertcaption%
%\end{beamercolorbox}%
%}

\definecolor{quotemark}{gray}{0.7}
\newenvironment{fancyquote}%
	{%
	    \vspace{1em}%
	    \singlespacing
	    \noindent%
		 \begin{picture}(0,0)%
		 \put(-15,-0){\makebox(0,0){\scalebox{3}{\textcolor{quotemark}{``}}}}%
		 \end{picture}%
	\footnotesize\upshape%
	}%
	{%
	 \par%
	 \makebox[0pt][l]{%
	 \hspace{\linewidth}%
	 \begin{picture}(0,0)(0,0)%
	 \put(15,20){\makebox(0,0){%
	 \scalebox{3}{\color{quotemark}''}}}%
	 \end{picture}}%
	   \vspace{-2.5em}%
	}%


\usepackage{diagbox} % \backslashbox in tables
