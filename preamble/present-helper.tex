%\usepackage[small,sf,bf]{titlesec}
\usepackage[utf8]{inputenc}
\usepackage[british]{babel}

\usepackage[table,x11names,dvipsnames,rgb]{xcolor}
\usepackage{tikz}
\usetikzlibrary{snakes,arrows,shapes,automata}
\usetikzlibrary{positioning}

\usepackage{float}
\usepackage{graphicx}

%% use as
%%     \vertcenterimage{\includegraphics{*}}
\newcommand{\vertcenterimage}[1]{\raisebox{-.5\height}{#1}}

%% use as
%%     \flipbox{\includegraphics{*}}
\newcommand{\flipbox}[1]{\scalebox{1}[-1]{#1}}


\usetheme{Ilmenau}
\usecolortheme{beaver}
\usepackage{ifxetex}

\ifxetex
	% set font to Tahoma
	\usefonttheme{professionalfonts} % using non standard fonts for beamer
	\usefonttheme{serif} % default family is serif
	\usepackage{fontspec}
	\setmainfont{Tahoma}
\else
	\usepackage[T1]{fontenc}
\fi

\definecolor{CBLRed}{RGB}{205,0,0}

% Title on title slide
\setbeamerfont{title}{size = \Large}
\setbeamercolor{title}{fg = black, bg = white}

\setbeamercolor{frametitle}{fg=white,bg=CBLRed}
\setbeamercolor{section in head/foot}{fg=white,bg=CBLRed}
\setbeamercolor{subsection in head/foot}{fg=white,bg=CBLRed}
\setbeamercolor{author in head/foot}{fg=white,bg=CBLRed}
\setbeamertemplate{headline}{}
\beamertemplatenavigationsymbolsempty

\logo{}
\def\insertlogo
{%
	\color{gray}
	\begin{tabular*}{\paperwidth}{p{0.01\paperwidth}p{0.84\paperwidth}p{0.15\paperwidth}}
		&
		\parbox[c][][c]{\textwidth}{\raggedright
			\vertcenterimage{\includegraphics[width=\textwidth]{gfx/cbl-footer.png}}
		}
		&
		\parbox[c][][c]{\textwidth}{%\raggedleft
			{\large\insertframenumber} %/\inserttotalframenumber
		}
	\end{tabular*}
 }

%\usefonttheme[onlymath]{serif}
\setbeamertemplate{frametitle}[default][center]
\setbeamertemplate{footline}{}
%\setbeamertemplate{footline}[frame number]

%% Change beamer bullets to circles rather than the ball default
\setbeamertemplate{itemize items}[circle]
\setbeamertemplate{enumerate items}[circle]


\usepackage{textcomp}
\usepackage{fancyvrb}
\usepackage{changepage}
\usepackage{multicol}
\usepackage{wasysym}
\usepackage{listings}

\lstset{%basicstyle=\small\ttfamily,
%numbers=left,
%escapeinside=||
}
\newenvironment{indented}{\begin{adjustwidth}{1.5em}{}}{\end{adjustwidth}}

% http://tex.stackexchange.com/questions/12550/changing-default-width-of-blocks-in-beamer/12551#12551
\newenvironment<>{varblock}[2][.9\textwidth]{%
  \setlength{\textwidth}{#1}
  \begin{actionenv}#3%
    \def\insertblocktitle{#2}%
    \par%
    \usebeamertemplate{block begin}}
  {\par%
    \usebeamertemplate{block end}%
  \end{actionenv}}


%% TOC at beginning of sections/subsections
%\AtBeginSection[]
%{
%% <beamer>{Table of Contents}
%        \begin{frame}
%                \tableofcontents[
%                                currentsection,
%                                hideothersubsections,
%                                sectionstyle=show/shaded,
%                                subsectionstyle=hide/hide,
%                ]
%        \end{frame}
%}
%\AtBeginSubsection[]
%{
%   \begin{frame}
%                \begin{tabular}{p{0.3\paperwidth}p{0.7\paperwidth}}
%                        \tableofcontents[
%                                        currentsection,
%                                        hideothersubsections,
%                                        sectionstyle=show/shaded,
%                                        subsectionstyle=hide/hide,
%                        ]
%                &
%                        \begin{varblock}[0.5\paperwidth]{}
%                                \vspace{.2cm}
%                                \tableofcontents[
%                                        currentsubsection,
%                                        hideothersubsections,
%                                        sectionstyle=hide/hide,
%                                        subsectionstyle=show/shaded,
%                                ]
%                        \end{varblock}
%                \end{tabular}
%   \end{frame}
%}


\AtBeginSubsection[]
{
	\begin{frame}
		\begin{multicols}{2}
			\tableofcontents[
				currentsection,
				hideothersubsections,
				sectionstyle=show/shaded,
				subsectionstyle=show/shaded,
			]
		\end{multicols}
  \end{frame}
}

%% TikZ arrows
%% From <https://tex.stackexchange.com/questions/61507/drawing-arrows-in-beamer>
\tikzset{
    myarrow/.style={
        draw,
        fill=orange,
        single arrow,
        minimum height=3.5ex,
        single arrow head extend=1ex
    }
}
\newcommand{\arrowup}{%
\tikz [baseline=-0.5ex]{\node [myarrow,rotate=90] {};}
}
\newcommand{\arrowdown}{%
\tikz [baseline=-1ex]{\node [myarrow,rotate=-90] {};}
}
\newcommand{\arrowright}{%
\tikz [baseline=-0.5ex]{\node [myarrow,rotate=0] {};}
}
\newcommand{\arrowleft}{%
\tikz [baseline=-0.5ex]{\node [myarrow,rotate=180] {};}
}
