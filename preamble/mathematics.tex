%% Mathematical commands
\newcommand{\mnth}[1]{#1^{\mathrm{th}}}
\newcommand{\freqdom}[1]{\widehat{#1}}
\newcommand{\Convolve}{\mathop{\ast}}%
\newcommand{\HadamardProd}{\mathop{\odot}}%
\newcommand{\SetIntersection}{\mathbin{\cap}}%

\newcommand{\FuncTo}[3]{#1\colon #2 \to #3}%
%% \FuncTo{f}{\Sreal^n}{\Sreal}
%% gives
%% f: R^n -> R

\newcommand{\FourierTrans}[1]{\ensuremath{\mathcal{F}\left\{#1\right\}}}
\newcommand{\IFourierTrans}[1]{\ensuremath{\mathcal{F}^{-1}\left\{#1\right\}}}

\DeclareMathOperator*{\argmin}{arg\,min}
\DeclareMathOperator*{\argmax}{arg\,max}

\newcommand{\Dim}[1]{\ensuremath{\left|#1\right|}}

%% Sets
\newcommand{\Sreal}{\ensuremath{\mathbb{R}}}
\newcommand{\Scomplex}{\ensuremath{\mathbb{C}}}
\newcommand{\Snatural}{\ensuremath{\mathbb{N}}}
\newcommand{\Sinteger}{\ensuremath{\mathbb{Z}}}
\newcommand{\Srational}{\ensuremath{\mathbb{Q}}}
\newcommand{\Sirrational}{\ensuremath{\mathbb{J}}}

%% Complex numbers
\newcommand{\realpart}[1]{\ensuremath{\mathfrak{R}\left\{#1\right\}}}
\newcommand{\imagpart}[1]{\ensuremath{\mathfrak{I}\left\{#1\right\}}}

\newcommand{\tens}[1]{\underline{#1}}            % tensor
\newcommand{\vect}[1]{\overrightarrow{#1}}       % vector
\newcommand{\tuple}[1]{\overline{#1}} % tuple
%\newcommand{\tuple}[1]{\bm{#1}} % tuple
%\newcommand{\tuple}[1]{\mathbf{#1}} % tuple
%\newcommand{\tuple}[1]{\boldsymbol{\mathbf{#1}}}
%\newcommand{\tuple}[1]{\boldsymbol{\mathbf{\bm{#1}}}}
%\newcommand*\tuple[1]{\mathbf{\bm{#1}}}
%\def\tuple#1{\bm{#1}}
\newcommand{\scalar}[1]{#1}                      % scalar

%\usepackage{bm,xstring}
%\def\tuple#1%
    %{\IfSubStr{ABCDEFGHIJKLMNOPQRSTUVWXYZabcdefghijklmnopqrstuvwxyz}{#1}
        %{\mathbf{#1}}
        %{\bm{#1}}}

\def\abs#1{\left| #1 \right|}	% use instead of $|x|$
\def\norm#1{\left\| #1 \right\|}% use instead of $\|x\|$
