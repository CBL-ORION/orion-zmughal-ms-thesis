\begin{savequote}[0.55\linewidth]
	\begin{fancyquote}
		Measure twice and cut once.
	\end{fancyquote}
	\qauthor{English proverb}

	\begin{fancyquote}
		But cut\hspace{\fill}ting is more fun than m\quad{}eas\quad{}uri\quad{}ng!
	\end{fancyquote}
	\qauthor{Anonymous}
\end{savequote}

\chapter{Analysis and requirements}\label{ch:analysis}
% TODO
% 2. Analysis and Requirements
%    [ Analysis of project requirements. How is the older codebase
%      structured. How is refactoring accomplished. ]

TODO TODO TODO
TODO TODO TODO
TODO TODO TODO
TODO TODO TODO
TODO TODO TODO
TODO TODO TODO
TODO TODO TODO
TODO TODO TODO
TODO TODO TODO
TODO TODO TODO
TODO TODO TODO
TODO TODO TODO
TODO TODO TODO
TODO TODO TODO
TODO TODO TODO
TODO TODO TODO
TODO TODO TODO
TODO TODO TODO
TODO TODO TODO
TODO TODO TODO
TODO TODO TODO

\section{Requirements}

\section{Challenges and risks}

\section{MATLAB version of ORION 3}

The or

\subsection{Callgraph}

In order to understand how the ORION MATLAB code is structured, it
is necessary to first get the call graph of the code. There is a
tool built in to MATLAB to do this called
depfun
\url{http://www.mathworks.com/help/matlab/ref/depfun.html} % TODO turn into citation
, however this tool runs slowly when running on the entire codebase.
There is an alternative called fdep
\url{http://www.mathworks.com/matlabcentral/fileexchange/17291-fdep--a-pedestrian-function-dependencies-finder}

The output of this tool is graph that % TODO

\input{02_analysis/call-graph.tex}
