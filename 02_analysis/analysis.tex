\begin{savequote}[0.55\linewidth]
	\begin{fancyquote}
		Measure twice and cut once.
	\end{fancyquote}
	\qauthor{Carpentry rule of thumb}

	\begin{fancyquote}
		But cut\hspace{1em}ting is more fun than m\hspace{0.5em}\raisebox{-0.1em}{eas}\hspace{0.5em}\raisebox{-0.3em}{uri}\hspace{0.5em}\raisebox{-0.5em}{ng!}
	\end{fancyquote}
	\qauthor{Anonymous}
\end{savequote}

\chapter{Analysis and requirements}\label{ch:analysis}
% TODO
% 2. Analysis and Requirements
%    [ Analysis of project requirements. How is the older codebase
%      structured. How is refactoring accomplished. ]

Before writing a single line of code, it is a good practice to understand the
scope of the problem. Gathering the deliverables and requirements of the
project is necessary not only for understanding how long the project will take,
but also the order in which to implement each component. Establishing this
order is very useful for test-driven development.
% TODO describe TDD

\section{Deliverables}\label{sec:deliverables}

The foremost deliverable is a complete conversion of the existing
code from MATLAB to native code. This involves an analysis of the
existing code to see where parts of the algorithm that can be
converted.

% TODO expand using the abstract


\section{Challenges and risks}

While the code for the algorithm already exists, starting with a line-by-line
translation of the MATLAB code has some limitations outlined as follows.
\begin{description}
\item[Toolbox\label{desc:matlab:toolbox}] The code is written to use MATLAB's extensive 
	specialised toolboxes for image processing and
	statistics which means that equivalents must be
	incorporated into the new codebase.
\item[Memory management\label{desc:matlab:mem}] Since MATLAB is a dynamic array language with
	automatic memory management, it is simple to create
	multidimensional array and extend it without having to
	keep track of the variable's size or the variable
	lifetime. Since C uses manual memory management, it is
	necessary to manually allocate and release memory to avoid
	memory leaks.
\item[Caching\label{desc:matlab:cache}] The ORION MATLAB code makes frequent use of the file system
	to cache calculations between runs. The purpose of this is
	to speed up experiments so that when an experiment is
	rerun, any images that have been processed in an earlier
	stage (i.e., segmentation) do not need to be reprocessed
	in later stages (i.e., centreline extraction). Code
	written in this form imposes an algorithm structure
	that is no longer strictly imperative --- the code is now
	interspersed with checks to see if the data already exists
	and instead of passing the data between functions using
	multidimensional arrays as parameters, the parameters to
	the functions are filenames.
\item[Subvolume\label{desc:matlab:subvol}] The MATLAB code breaks up the input data into
	subvolumes. This allows the computation to run a small
	region of the data which allows for processing data that
	may be too large to fit entirely memory. Furthermore, when
	used in conjunction with the aforementioned caching, the
	steps used for each processing stage can be more granular
	which means that if any processing is incomplete (e.g.,
	because the computer runs out of memory or disk space),
	the data is not entirely lost. However, this complicates
	the algorithm because any calculation involving
	coordinates in a volume must map indices in subvolumes to
	indices in the corresponding supervolume.
\end{description}

The \nameref{desc:matlab:cache} and \nameref{desc:matlab:subvol} issues both
indicate issues that can be described as cross-cutting concerns. Cross-cutting
concerns are  % TODO aspects ...

% TODO expand the below

This project has

\begin{itemize}
\item starting with no tests
\item no automated build system
\item works entirely inside of MATLAB
\item has executable components that are called via \texttt{system()} which
	means that they need to be compiled and the path to the components
	needs to be set
\end{itemize}

\section{MATLAB version of ORION 3}


\subsection{Callgraph}

In order to understand how the ORION MATLAB code is structured, it
is necessary to first get the call graph of the code. There is a
tool built in to MATLAB to do this called
depfun
\url{http://www.mathworks.com/help/matlab/ref/depfun.html} % TODO turn into citation
, however this tool runs slowly when running on the entire codebase.
There is an alternative called fdep
\url{http://www.mathworks.com/matlabcentral/fileexchange/17291-fdep--a-pedestrian-function-dependencies-finder}

The output of this tool is graph that % TODO


\begin{tikzpicture}[node distance=0.3cm]

	\tikzstyle{process} = [draw,shape=rectangle,scale=0.3]
	\tikzstyle{data} = [draw,shape=circle,style=dashed,scale=0.3]
	%
	\node (data_vol)        at (0,0)              [data] {$\mathrm{3D~volume}$};

	\node (proc_seg)        [right=of data_vol] [process] {$\mathrm{Segmentation}$};
	\node (proc_reg)        [right=of proc_seg] [process] {$\mathrm{Registration}$};
	\node (proc_trace)      [right=of proc_reg] [process] {$\mathrm{Tracing}$};

	\node (data_centerline) [right=of proc_trace] [data] {$\mathrm{Centerline}$};

	\draw [->] (data_vol)   --  (proc_seg);
	\draw [->] (proc_seg)   --  (proc_reg);
	\draw [->] (proc_reg)   --  (proc_trace);
	\draw [->] (proc_trace) --  (data_centreline);
%
\end{tikzpicture}



\resizebox{1.0\textwidth}{!}{\begin{tikzpicture}[node distance=0.5cm]
	\tikzstyle{step} = [draw,shape=circle,scale=0.4];
	%
	\node (step_bg_train)        at (0,0)             [draw,step] {\parbox{3cm}{\centering{}Obtain background training data}};

	\node (step_feat)        [right=of step_bg_train] [draw,step] {\parbox{2cm}{\centering{}Obtain vessellness features}};
	\node (step_discrim)     [right=of step_feat]     [draw,step] {\parbox{2cm}{\centering{}Learn discriminant function}};
	\node (step_thresh)      [below=of step_discrim]  [draw,step] {\parbox{2cm}{\centering{}Threshold}};
	\node (step_post)        [left=of step_thresh]    [draw,step] {\parbox{2cm}{\centering{}Post-processing}};

	\draw [->] (step_bg_train) -- (step_feat);
	\draw [->] (step_feat) -- (step_discrim);
	\draw [->] (step_discrim) -- (step_thresh);
	\draw [->] (step_thresh) -- (step_post);
%
\end{tikzpicture}
}

\resizebox{1.0\textwidth}{!}{\input{gfx-out/matlab-call-graph/registration.tex}}

\resizebox{1.0\textwidth}{!}{\begin{tikzpicture}[node distance=0.5cm]
	\tikzstyle{step} = [draw,shape=circle,scale=0.4];
	%
	\node (step_shortest_path)        at (0,0)             [draw,step] {\parbox{3cm}{\centering{}Shortest path using fast-marching}};

	\node (step_centre_pts)          [right=of step_shortest_path]    [draw,step] {\parbox{2cm}{\centering{}Centre of dendrite}};
	\node (step_radii_estimation)    [right=of step_centre_pts]       [draw,step] {\parbox{2cm}{\centering{}Radii estimation}};
	\node (step_neumorph_repr)       [below=of step_radii_estimation] [draw,step] {\parbox{3cm}{\centering{}Neuron morphology representation}};
	\node (step_post)                [left=of step_neumorph_repr]    [draw,step] {\parbox{2cm}{\centering{}Post-processing of tracing}};

	\draw [->] (step_shortest_path)    -- (step_centre_pts);
	\draw [->] (step_centre_pts)       -- (step_radii_estimation);
	\draw [->] (step_radii_estimation) -- (step_neumorph_repr);
	\draw [->] (step_neumorph_repr)    -- (step_post);
%
\end{tikzpicture}
}

%\input{02_analysis/call-graph.tex}

\subsection{Processing pipeline}

\subsection{Algorithm}


