\documentclass{beamer}
\usepackage[utf8]{inputenc}
\usepackage{tikz}
%\usepackage[small,sf,bf]{titlesec}
\usepackage[utf8]{inputenc}
\usepackage[british]{babel}

\usepackage[table,x11names,dvipsnames,rgb]{xcolor}
\usepackage{tikz}
\usetikzlibrary{snakes,arrows,shapes,automata}
\usetikzlibrary{positioning}

\usepackage{float}
\usepackage{graphicx}

%% use as
%%     \vertcenterimage{\includegraphics{*}}
\newcommand{\vertcenterimage}[1]{\raisebox{-.5\height}{#1}}

%% use as
%%     \flipbox{\includegraphics{*}}
\newcommand{\flipbox}[1]{\scalebox{1}[-1]{#1}}


\usetheme{Ilmenau}
\usecolortheme{beaver}
\usepackage{ifxetex}

\ifxetex
	% set font to Tahoma
	\usefonttheme{professionalfonts} % using non standard fonts for beamer
	\usefonttheme{serif} % default family is serif
	\usepackage{fontspec}
	\setmainfont{Tahoma}
\else
	\usepackage[T1]{fontenc}
\fi

\definecolor{CBLRed}{RGB}{205,0,0}

% Title on title slide
\setbeamerfont{title}{size = \Large}
\setbeamercolor{title}{fg = black, bg = white}

\setbeamercolor{frametitle}{fg=white,bg=CBLRed}
\setbeamercolor{section in head/foot}{fg=white,bg=CBLRed}
\setbeamercolor{subsection in head/foot}{fg=white,bg=CBLRed}
\setbeamercolor{author in head/foot}{fg=white,bg=CBLRed}
\setbeamertemplate{headline}{}
\beamertemplatenavigationsymbolsempty

\logo{}
\def\insertlogo
{%
	\color{gray}
	\begin{tabular*}{\paperwidth}{p{0.01\paperwidth}p{0.84\paperwidth}p{0.15\paperwidth}}
		&
		\parbox[c][][c]{\textwidth}{\raggedright
			\vertcenterimage{\includegraphics[width=\textwidth]{gfx/cbl-footer.png}}
		}
		&
		\parbox[c][][c]{\textwidth}{%\raggedleft
			{\large\insertframenumber} %/\inserttotalframenumber
		}
	\end{tabular*}
 }

%\usefonttheme[onlymath]{serif}
\setbeamertemplate{frametitle}[default][center]
\setbeamertemplate{footline}{}
%\setbeamertemplate{footline}[frame number]

%% Change beamer bullets to circles rather than the ball default
\setbeamertemplate{itemize items}[circle]
\setbeamertemplate{enumerate items}[circle]


\usepackage{textcomp}
\usepackage{fancyvrb}
\usepackage{changepage}
\usepackage{multicol}
\usepackage{wasysym}
\usepackage{listings}

\lstset{%basicstyle=\small\ttfamily,
%numbers=left,
%escapeinside=||
}
\newenvironment{indented}{\begin{adjustwidth}{1.5em}{}}{\end{adjustwidth}}

% http://tex.stackexchange.com/questions/12550/changing-default-width-of-blocks-in-beamer/12551#12551
\newenvironment<>{varblock}[2][.9\textwidth]{%
  \setlength{\textwidth}{#1}
  \begin{actionenv}#3%
    \def\insertblocktitle{#2}%
    \par%
    \usebeamertemplate{block begin}}
  {\par%
    \usebeamertemplate{block end}%
  \end{actionenv}}


%% TOC at beginning of sections/subsections
%\AtBeginSection[]
%{
%% <beamer>{Table of Contents}
%        \begin{frame}
%                \tableofcontents[
%                                currentsection,
%                                hideothersubsections,
%                                sectionstyle=show/shaded,
%                                subsectionstyle=hide/hide,
%                ]
%        \end{frame}
%}
%\AtBeginSubsection[]
%{
%   \begin{frame}
%                \begin{tabular}{p{0.3\paperwidth}p{0.7\paperwidth}}
%                        \tableofcontents[
%                                        currentsection,
%                                        hideothersubsections,
%                                        sectionstyle=show/shaded,
%                                        subsectionstyle=hide/hide,
%                        ]
%                &
%                        \begin{varblock}[0.5\paperwidth]{}
%                                \vspace{.2cm}
%                                \tableofcontents[
%                                        currentsubsection,
%                                        hideothersubsections,
%                                        sectionstyle=hide/hide,
%                                        subsectionstyle=show/shaded,
%                                ]
%                        \end{varblock}
%                \end{tabular}
%   \end{frame}
%}


\AtBeginSubsection[]
{
	\begin{frame}
		\begin{multicols}{2}
			\tableofcontents[
				currentsection,
				hideothersubsections,
				sectionstyle=show/shaded,
				subsectionstyle=show/shaded,
			]
		\end{multicols}
  \end{frame}
}

%% TikZ arrows
%% From <https://tex.stackexchange.com/questions/61507/drawing-arrows-in-beamer>
\tikzset{
    myarrow/.style={
        draw,
        fill=orange,
        single arrow,
        minimum height=3.5ex,
        single arrow head extend=1ex
    }
}
\newcommand{\arrowup}{%
\tikz [baseline=-0.5ex]{\node [myarrow,rotate=90] {};}
}
\newcommand{\arrowdown}{%
\tikz [baseline=-1ex]{\node [myarrow,rotate=-90] {};}
}
\newcommand{\arrowright}{%
\tikz [baseline=-0.5ex]{\node [myarrow,rotate=0] {};}
}
\newcommand{\arrowleft}{%
\tikz [baseline=-0.5ex]{\node [myarrow,rotate=180] {};}
}

\usetikzlibrary{mindmap,trees,shadows}

%\usetikzlibrary{mindmap,trees,shadows}

\tikzset{
    vertex/.style = {
        circle,
        fill            = black,
        outer sep = 2pt,
        inner sep = 1pt,
    }
}

\newcommand{\myfun}[2] {$fun^{#1}_{#2}$}


%% tikz parameters
\usetikzlibrary{shapes }
% Styles 
\tikzstyle{function} = [draw,fill=blue,shape=rectangle, text centered,text=white, text width=3cm, minimum size=2.5cm]
\tikzstyle{data}     = [draw,fill=orange,shape=ellipse, text centered,text=white, text width=1.5cm, minimum size=2.5cm]
\tikzstyle{f2d}  = [line width=0.1cm,draw=pink]
\tikzstyle{d2f}  = [line width=0.1cm,draw=green]



\DeclareMathOperator*{\argmin}{arg\,min}
\DeclareMathOperator*{\argmax}{arg\,max}


\begin{document}

\newcommand{\InputVolumeIndices}{\ensuremath{[i,j,k]}}
\newcommand{\Dim}[1]{\ensuremath{\left|#1\right|}}
\newcommand{\RadiiScalesName}{\ensuremath{r}}
 

\begin{frame}[fragile]
\frametitle{Charakteristika}
 %% Mathematical commands
\resizebox{1.0\textwidth}{!}{
\begin{tikzpicture}
  \tikzset{
    invisible/.style={opacity=0},
    transparent/.style={opacity=0.5},
    visible/.style={opacity=1},
    visible on/.style={temporal=#1{invisible}{visible}{transparent}},
    temporal/.code args={<#1>#2#3#4}{%
      \temporal<#1>{\pgfkeysalso{#2}}{\pgfkeysalso{#3}}{\pgfkeysalso{#4}
      } % \pgfkeysalso doesn't change the path
    },
  }
%% Input nodes
\node[data, label={[align=left] below:Scalar Degree of Taylor}] (dl) at (0,8) {$d^{L}$};  
\node[data, label={[align=left] below:$M = m_{0}  \times m_{1} \times m_{2} $\\ Volume of \\ intensity pixels}] (Vijk) at (0,4){$V\InputVolumeIndices$};  
\node[data, label={[align=left] below:$n$-vector\\ with elements\\ representing\\ the radius to\\ segment (scale)}] (rs) at (0,0) {$r_{s}$};  

%% First row nodes
\node[function, visible on=<2>] (field21) at ( 4,4) {Laplacian\\ scale\\ parameter};
\node[data, visible on=<3>, label={[align=left, visible on=<3>] below:$\Dim{\RadiiScalesName}-vector$}] (field22) at ( 8,4) {$\sigma^{G}_{s}$};
\node[function, visible on=<4>] (field23) at (12,4) {Laplacian\\ filter\\ (HDAF)};
\node[data, visible on=<5>] (field24) at (16,4) {$V^{L}_{s}\InputVolumeIndices$};
\node[function, visible on=<6>] (field25) at (20,4) {$\argmax_{s}V^{L}_{s}\InputVolumeIndices$};

\node[data, visible on=<7>, label={[align=left, visible on=<7>] below:M volume of $s^{th}$\\  index elements}] (field26) at (24,4) {$R_{\mathrm{max}}^{L}\InputVolumeIndices$}; 

  
%% Second row nodes
\node[function, visible on=<8>] (field11) at ( 4,0) {Hessian\\ scale\\ parameter};
\node[data, visible on=<9>,  label={[align=left, visible on=<9>] below:$\Dim{\RadiiScalesName}-vector$}] (field12) at ( 8,0) {$O^{G}_{s}$};
\node[function, visible on=<10>] (field13) at (12,0) {Hessian\\ filter};
\node[data, visible on=<11>,  label={[align=left, visible on=<11>] below:$\Dim{\RadiiScalesName}-vector$}] (field14) at (16,0) {$ \lambda_{s,1}\InputVolumeIndices$\\ $\lambda_{s,2}\InputVolumeIndices$\\ $\lambda_{s,3}\InputVolumeIndices$\\};
\node[function, visible on=<12>] (field15) at (20,0) {$F^{G}\InputVolumeIndices$};


%% First row connections
\draw[->,d2f, visible on=<4>] (dl) to[in=90,out=0] (field23);
\draw[->,d2f, visible on=<4>] (Vijk) to[in=150,out=45] (field23);
\draw[->,d2f, visible on=<10>] (Vijk) to[in=150,out=325] (field13);
\draw[->,d2f, visible on=<2>] (rs) to[in=180,out=0] (field21);
\draw[->,d2f, visible on=<8>] (rs) to[in=180,out=0] (field11);

%% L1
\draw[->,f2d, visible on=<9>] (field11) to[in=180,out=0] (field12);
\draw[->,d2f, visible on=<10>] (field12) to[in=180,out=0] (field13);
\draw[->,f2d, visible on=<11>] (field13) to[in=180,out=0] (field14);
\draw[->,d2f, visible on=<12>] (field14) to[in=180,out=0] (field15);

%% L2
\draw[->,f2d, visible on=<3>] (field21) to[in=180,out=0] (field22);
\draw[->,d2f, visible on=<4>] (field22) to[in=180,out=0] (field23);
\draw[->,f2d, visible on=<5>] (field23) to[in=180,out=0] (field24);
\draw[->,d2f, visible on=<6>] (field24) to[in=180,out=0] (field25);
\draw[->,f2d, visible on=<7>] (field25) to[in=180,out=0] (field26);
\end{tikzpicture} 
}
\end{frame}

\end{document}