% 350-word limit
\begin{abstract}

% * Problem - what is being solved?
The thesis describes the conversion of the \acrfull{ORION} system
for neuron-morphology reconstruction from an interpreted language to a compiled
language. The motivation of this conversion is to provide a tool
that can be used by neuroscience researchers to analyze their own
neuron data and compare the output against both manual and
automated tracings. This is in line with the goals of open
science: a movement that seeks to make the findings and processes
of research more widely available for peer review and
reproducibility. By collaboratively sharing both neuron-imaging
data and code between organizations, it is possible to compare the
results of multiple methods without reimplementing all the stages
of the reconstruction pipeline.

% * Challenges - what are the difficulties of the problem?
In order to release the existing algorithm so that it can easily
be incorporated into other tools, the implementation must be
rewritten in a different language. This presents a challenge
because the languages have vastly different paradigms. As a result,
much of the existing code needs to be analyzed to determine any
changes needed to the design. Creating a new implementation also means
that the new system can be designed with modifiability in mind so
that future changes can be easily incorporated.
%
% * Objectives - what were the main goals/hypotheses?
The specific objectives are to
\begin{enumerate*}[label=(\roman*)]
\item analyze the ORION algorithm and implementation to
	determine the architecture for the new system that is
	efficient and extensible;
\item integrate the system into a popular toolkit for biomedical
	image analysis for ease-of-use and visualization;
\item develop a test suite of both the individual components (unit
	testing) and across the whole system (integration tests);
	and % "and" before last item
\item ensure that the software gives reproducible results by
	making it easy to build and distribute. % end of sentence
\end{enumerate*}

% * Impact - how does this effect the scientific community?
The reconstruction of neuron morphology from microscopy imaging data
is an invaluable method for understanding neuron characteristics.
However, due to the cost in time and effort, manual neuron
reconstruction is not feasible for large-scale analysis of neuron
datasets.
% * Methods - how were the objectives pursued?
% * Results - what were the (important) findings?
% * Conclusions - what does it all mean?
This implementation provides a working method for determining
neuron morphology that can be used to collect statistical
properties from various neuron data
that can also be extended by the community.

\end{abstract}
	\centerline{\vrule width 1.5in height 0.4pt}
\noindent\textbf{Keywords}: neurons, cell morphology, biomedical image analysis, software engineering
