\newcommand{\SegGroundTotal}{S_{\mathrm{G,T}}}
\newcommand{\SegAutomaticTotal}{S_{\mathrm{A,T}}}
\newcommand{\SegAutomaticCorrect}{S_{\mathrm{A,C}}}
\newcommand{\SegAutomaticMissing}{S_{\mathrm{A,miss}}}
\newcommand{\SegAutomaticExtra}{S_{\mathrm{A,extra}}}

\newcommand{\InputVolumeIndices}{[i,j,k]}
\newcommand{\ScaleIndex}{s}

%% Input to the segmentation
\newcommand{\InputVolumeName}{\tens{V}}
\newcommand{\InputVolumeElemIndices}{\InputVolume\InputVolumeIndices}
\newcommand{\InputVolumeDimensions}{\Dim{\InputVolumeName}_0 \times \Dim{\InputVolumeName}_1 \times \Dim{\InputVolumeName}_2}

%% Used by the Laplacian filter
\newcommand{\LaplacianFilterApproxDegree}{\scalar{d^{L}}}
\newcommand{\LaplacianFilterScaleFactor}{\scalar{\xi^{L}}}

%% Used to constrol the \LaplacianScales and \HessianScales
\newcommand{\RadiiScalesName}{\tuple{r}}
\newcommand{\RadiiScalesElem}{\scalar{r}_{\ScaleIndex}}

%% Input to Laplacian filter
\newcommand{\LaplacianScalesName}{\tuple{\sigma}^{L}}
\newcommand{\LaplacianScalesElem}{\scalar{\sigma}^{L}_{\ScaleIndex}}

%% Input to Hessian filter
\newcommand{\HessianScalesName}{\tuple{\sigma}^{G}}
\newcommand{\HessianScalesElem}{\scalar{\sigma}^{G}_{\ScaleIndex}}

%% Output of the Laplacian filter
\newcommand{\LaplacianOutputVolumeName}{\tuple{V}^{L}}
\newcommand{\LaplacianOutputVolumeElem}{\tens{V}^{L}_{\ScaleIndex}}
\newcommand{\LaplacianOutputVolumeElemIndices}{\LaplacianOutputVolumeElem\InputVolumeIndices}

%% Eigenvalue 1,2,3
%% 1
\newcommand{\EigFirstName}{\tuple{\lambda}_{1}}
\newcommand{\EigFirstElem}{\tens{\lambda}_{s,1}}
\newcommand{\EigFirstElemIndices}{\EigFirstElem\InputVolumeIndices}
%% 2
\newcommand{\EigSecondName}{\tuple{\lambda}_{2}}
\newcommand{\EigSecondElem}{\tens{\lambda}_{s,2}}
\newcommand{\EigSecondElemIndices}{\EigSecondElem\InputVolumeIndices}
%% 3
\newcommand{\EigThirdName}{\tuple{\lambda}_{3}}
\newcommand{\EigThirdElem}{\tens{\lambda}_{s,3}}
\newcommand{\EigThirdElemIndices}{\EigThirdElem\InputVolumeIndices}

\newcommand{\MaxResponseLaplacianName}{\tens{R}_{\mathrm{max}}^{L}}
\newcommand{\MaxResponseLaplacianElemIndices}{\MaxResponseLaplacianName\InputVolumeIndices}

\newcommand{\FeaturesName}{\tens{F}^{G}}
\newcommand{\FeaturesElemIndices}{\FeaturesName\InputVolumeIndices}

\newcommand{\HistogramBinCount}{\scalar{H}_{n}}

\newcommand{\FeatureIndex}{q}
\newcommand{\HistogramMinElem}{\scalar{h}_{\mathrm{min},\FeatureIndex}}
\newcommand{\HistogramMaxElem}{\scalar{h}_{\mathrm{max},\FeatureIndex}}

% TODO Notation and stylistic conventions

\newacronym{SDLC}{SDLC}{Systems Development Life Cycle}

\newacronym{FFT}{FFT}{Fast-Fourier transform}
\newacronym{ORION}{ORION}{Online Reconstruction and functional Imaging Of Neurons}
\newacronym{API}{API}{Application Programming Interface}
\newacronym{ABI}{ABI}{Application Binary Interface}
\newacronym{DIADEM}{DIADEM}{Digital Reconstruction of Axonal and Dendritic Morphology}

\newacronym{SVM}{SVM}{Support Vector Machine}

\newacronym{EEG}{EEG}{Electroencephalography}
\newacronym{fMRI}{fMRI}{Functional Magnetic Resonance Imaging}
\newacronym{NITRC}{NITRC}{Neuroimaging Informatics Tools and Resources Clearinghouse}
\newacronym{NIF}{NIF}{Neuroscience Information Framework}

%% Symbols: symbols used in the text

\newglossaryentry{orionc}{%
	name={\ensuremath{\mathrm{{ORION}^{c}}}},%
	sort={sym: orionc},
	description={C version of \gls{orion3}}}

\newglossaryentry{orionmat}{%
	name={\ensuremath{\mathrm{{ORION}^{m}}}},%
	sort={sym: orionmat},
	description={MATLAB version of \gls{orion3}}}

\newglossaryentry{orion3}{%
	name={ORION3},%
	sort={sym: orion3},
	description={Version 3 of the \acrshort{ORION} algorithm described in~\autocite{ORION_Santamaria-Pang2015}}}

\newglossaryentry{orionmattoc}{%
	name={\ensuremath{\mathrm{{ORION}^{m\rightarrow{}c}}}},%
	sort={sym: orionmattoc},
	description={Test setup that uses the output of previous stages of \gls{orionmat} as input for a stage of \gls{orionc}}}

%% Notation: not used in the text

\newglossaryentry{not:Fourier}{%
	name={\ensuremath{       {\FourierTrans{\cdot}}}},%
	sort={not: fourier},
	description={Fourier transform of argument}}
\newglossaryentry{not:IFourier}{%
	name={\ensuremath{       {\IFourierTrans{\cdot}}}},%
	sort={not: fourier inverse},
	description={Inverse Fourier transform of argument}}

\newglossaryentry{not:FreqDomain}{%
	name={\ensuremath{       {\freqdom{x}}}},%
	sort={not: frequency domain},
	description={Frequency domain representation of signal $x$ such that $\freqdom{x} = \FourierTrans{x}$}}

\newglossaryentry{not:OpConvolv}{%
	name={\ensuremath{       {z = x \Convolve y}}},%
	sort={not: convolution},
	description={Convolution of $x$ and $y$}}
\newglossaryentry{not:OpMult}{%
	name={\ensuremath{       {z = x \HadamardProd y}}},%
	sort={not: product, Hadamard},
	description={Hadamard product (or pointwise/entrywise
product) of $\vect{x}$ and $\vect{y}$ such that $\vect{z}_i = \vect{x}_i \vect{y}_i$ }}



\makeglossaries

\setglossarystyle{cbl-gloss}
