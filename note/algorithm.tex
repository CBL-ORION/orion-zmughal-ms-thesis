\documentclass[12pt]{article}
\usepackage[utf8]{inputenc}
\usepackage[british]{babel}

\usepackage{amsmath}
\usepackage{bm} % bold mathematics (\bm command)
\usepackage{tocbibind}

\usepackage[table,x11names,rgb]{xcolor}
\usepackage{tikz}
\usetikzlibrary{snakes,arrows,shapes}
\usetikzlibrary{positioning}

\usepackage{hyperref}
\hypersetup{
    colorlinks,
    linkcolor={red!50!black},
    citecolor={blue!50!black},
    urlcolor={blue!80!black}
}

%% glossaries needs to be after hyperref
\usepackage[toc,acronym]{glossaries}
\glsdisablehyper % only link from the glossary to the text
\glsnopostdottrue % no period at the end of a glossary entry

\newcommand{\cblglossdelimiter}{:}
\newglossarystyle{cbl-gloss}{%
\setglossarystyle{list}% base this style on the list style
\renewcommand*{\glossentry}[2]{%
\item[\glsentryitem{##1}%
  \glstarget{##1}{\glossentryname{##1}\cblglossdelimiter}]
  \glossentrydesc{##1}\glspostdescription\space ##2}
}

%% For displaying algorithms using algorithmicx
\usepackage{algorithm}
\usepackage{algpseudocode}
%% make \listofalgorithms work with tocbibind.
%% From "packages algorithm and tocbibind" <http://newsgroups.derkeiler.com/Archive/Comp/comp.text.tex/2005-10/msg01099.html>
\makeatletter
\let\l@algorithm\l@figure
\makeatother
\renewcommand{\listofalgorithms}{\begingroup
  \tocfile{List of Algorithms}{loa}
\endgroup}


\usepackage{url}
\usepackage{longtable}
\usepackage{float}
\usepackage{graphicx}
\usepackage{mathtools}
\usepackage{multirow,booktabs}
%\usepackage[authoryear,sort,comma]{natbib}
%\newcommand{\autocite}[1]{\citep{#1}}
\usepackage[doublespacing]{setspace}
\usepackage{caption} % sets the captions to singlespacing
\captionsetup[figure]{labelfont=it}

% style=authoryear
\usepackage[%
	bibstyle=ieee,citestyle=numeric-comp,%
	sorting=none,backend=biber,%
	%maxcitenames=1,%
	urldate=long,%
	isbn=false,url=false % remove extra info
] {biblatex}

\usepackage[toc,page]{appendix}

\usepackage{siunitx}
\usepackage[inline]{enumitem}
% from <http://tex.stackexchange.com/questions/56249/enumitem-package-and-description-lists>
% use with enumitem like:
%     \begin{description}[font=\textpluscolon]
%         \item[A] ...
%         \item[B] ...
%     \end{description}
\newcommand*{\textpluscolon}[1]{{#1:}}

\usepackage{attrib}
\usepackage{listings}


%\usepackage{epsfig}
%\usepackage{epsf}

%%%%%%%%%%%%%%%%%%%%%%%%%%%%%%%%%%%%%%%%%%%%%%%%%% {{{
\usepackage{usebib}
%% prints out the info for a citation key:
%%     \printarticle{Author00}
\newcommand{\printarticle}[1]{\citeauthor{#1}, ``\usebibentry{#1}{title}''}
%%%%%%%%%%%%%%%%%%%%%%%%%%%%%%%%%%%%%%%%%%%%%%%%%% }}}

%% Fancy quote %%
%% adapted from <http://tex.stackexchange.com/questions/53377/inspirational-quote-at-start-of-chapter/53452#53452>
\usepackage{quotchap}
\definecolor{quotemark}{gray}{0.7}
\newenvironment{fancyquote}%
	{%
	    \vspace{1em}%
	    \singlespacing
	    \noindent%
		 \begin{picture}(0,0)%
		 \put(-15,-0){\makebox(0,0){\scalebox{3}{\textcolor{quotemark}{``}}}}%
		 \end{picture}%
	\footnotesize\upshape%
	}%
	{%
	 \par%
	 \makebox[0pt][l]{%
	 \hspace{\linewidth}%
	 \begin{picture}(0,0)(0,0)%
	 \put(15,20){\makebox(0,0){%
	 \scalebox{3}{\color{quotemark}''}}}%
	 \end{picture}}%
	   \vspace{-2.5em}%
	}%

%% Reference description environment
%% From <http://tex.stackexchange.com/questions/1230/reference-name-of-description-list-item-in-latex>
%%
%% Usage:
%%
%%     \begin{description}
%%         \item [Vehicle\label{itm:vehicle}] Something
%%         \item [Bus\label{itm:bus}] A type of \ref{itm:vehicle}
%%         \item [Car\label{itm:car}] A type of \ref{itm:vehicle} smaller than a \ref{itm:bus}
%%     \end{description}
%%
%%     The item `\ref{itm:bus}' is listed on page~\pageref{itm:bus} in section~\nameref{itm:bus}.

\usepackage{nameref}

\makeatletter
\let\orgdescriptionlabel\descriptionlabel
\renewcommand*{\descriptionlabel}[1]{%
  \let\orglabel\label
  \let\label\@gobble
  \phantomsection
  \edef\@currentlabel{#1}%
  %\edef\@currentlabelname{#1}%
  \let\label\orglabel
  \orgdescriptionlabel{#1}%
}
\makeatother


%% number biblatex bibliography section when the biblatex style is not
%% numeric (e.g., authoryear)
%\defbibenvironment{bibliography}
  %{\enumerate
	  %\singlespacing
     %{}
     %{\setlength{\leftmargin}{\bibhang}%
      %\setlength{\itemindent}{-\leftmargin}%
      %\setlength{\itemsep}{\bibitemsep}%
      %\setlength{\parsep}{\bibparsep}}}
  %{\endenumerate}
  %{\item}

%% Command used to create description like items inside an `enumerate` environment
%%
%% e.g., with enumitem's inline enumerate* environment:
%%
%%     \begin{enumerate*}[label={\alph*)}]
%%       \enumdescitem{First} example
%%       \enumdescitem{Second} example
%%       \enumdescitem{Third} example
%%     \end{enumerate*}
\newcommand\enumdescitem[1]{\item{\bfseries#1:\,}}

\usepackage[nameinlink,capitalize]{cleveref}

%% Mathematical commands
\newcommand{\freqdom}[1]{\widehat{#1}}
\newcommand{\Convolve}{\mathop{\ast}}%
\newcommand{\HadamardProd}{\mathop{\odot}}%

\newcommand{\FourierTrans}[1]{\ensuremath{\mathcal{F}\left\{#1\right\}}}
\newcommand{\IFourierTrans}[1]{\ensuremath{\mathcal{F}^{-1}\left\{#1\right\}}}

\newcommand{\SegGroundTotal}{S_{\mathrm{G,T}}}
\newcommand{\SegAutomaticTotal}{S_{\mathrm{A,T}}}
\newcommand{\SegAutomaticCorrect}{S_{\mathrm{A,C}}}
\newcommand{\SegAutomaticMissing}{S_{\mathrm{A,miss}}}
\newcommand{\SegAutomaticExtra}{S_{\mathrm{A,extra}}}

\newcommand{\InputVolumeIndices}{[i,j,k]}
\newcommand{\ScaleIndex}{s}

%%% Input to the segmentation
\newcommand{\InputVolumeName}{\tens{V}}
\newcommand{\InputVolume}{\InputVolumeName}
\newcommand{\InputVolumeElem}{\InputVolume\InputVolumeIndices}
\newcommand{\InputVolumeDimensions}{\Dim{\InputVolumeName}_0 \times \Dim{\InputVolumeName}_1 \times \Dim{\InputVolumeName}_2}

%%% Used by the Laplacian filter
\newcommand{\LaplacianFilterApproxDegree}{d^{L}}

%%% Used to constrol the \LaplacianScales and \HessianScales
\newcommand{\RadiiScalesName}{\vect{r}}
\newcommand{\RadiiScales}{\vect{r}_{\ScaleIndex}}

%%% Input to Laplacian filter
\newcommand{\LaplacianScalesName}{\vect{\sigma}^{L}}
\newcommand{\LaplacianScales}{\vect{\sigma}^{L}_{\ScaleIndex}}

%%% Input to Hessian filter
\newcommand{\HessianScalesName}{\vect{\sigma}^{G}}
\newcommand{\HessianScales}{\vect{\sigma}^{G}_{\ScaleIndex}}

%%% Output of the Laplacian filter
\newcommand{\LaplacianOutputVolumeName}{\tens{V}^{L}}
\newcommand{\LaplacianOutputVolume}{\tens{V}^{L}_{\ScaleIndex}}
\newcommand{\LaplacianOutputVolumeElem}{\LaplacianOutputVolume\InputVolumeIndices}

% TODO Notation and stylistic conventions

\newacronym{SDLC}{SDLC}{Systems Development Life Cycle}

\newacronym{FFT}{FFT}{Fast-Fourier transform}
\newacronym{ORION}{ORION}{Online Reconstruction and functional Imaging Of Neurons}
\newacronym{API}{API}{Application Programming Interface}
\newacronym{ABI}{ABI}{Application Binary Interface}
\newacronym{DIADEM}{DIADEM}{Digital Reconstruction of Axonal and Dendritic Morphology}

\newacronym{EEG}{EEG}{Electroencephalography}
\newacronym{fMRI}{fMRI}{Functional Magnetic Resonance Imaging}
\newacronym{NITRC}{NITRC}{Neuroimaging Informatics Tools and Resources Clearinghouse}
\newacronym{NIF}{NIF}{Neuroscience Information Framework}

%% Symbols: symbols used in the text

\newglossaryentry{orionc}{%
	name={\ensuremath{\mathrm{{ORION}^{c}}}},%
	sort={sym: orionc},
	description={C version of \gls{orion3}}}

\newglossaryentry{orionmat}{%
	name={\ensuremath{\mathrm{{ORION}^{m}}}},%
	sort={sym: orionmat},
	description={MATLAB version of \gls{orion3}}}

\newglossaryentry{orion3}{%
	name={ORION3},%
	sort={sym: orion3},
	description={Version 3 of the \acrshort{ORION} algorithm described in~\autocite{ORION_Santamaria-Pang2015}}}

%% Notation: not used in the text

\newglossaryentry{not:Fourier}{%
	name={\ensuremath{       {\FourierTrans{\cdot}}}},%
	sort={not: fourier},
	description={Fourier transform of argument}}
\newglossaryentry{not:IFourier}{%
	name={\ensuremath{       {\IFourierTrans{\cdot}}}},%
	sort={not: fourier inverse},
	description={Inverse Fourier transform of argument}}

\newglossaryentry{not:FreqDomain}{%
	name={\ensuremath{       {\freqdom{x}}}},%
	sort={not: frequency domain},
	description={Frequency domain representation of signal $x$ such that $\freqdom{x} = \FourierTrans{x}$}}

\newglossaryentry{not:OpConvolv}{%
	name={\ensuremath{       {z = x \Convolve y}}},%
	sort={not: convolution},
	description={Convolution of $x$ and $y$}}
\newglossaryentry{not:OpMult}{%
	name={\ensuremath{       {z = x \HadamardProd y}}},%
	sort={not: product, Hadamard},
	description={Hadamard product (or pointwise/entrywise
product) of $\vect{x}$ and $\vect{y}$ such that $\vect{z}_i = \vect{x}_i \vect{y}_i$ }}



\makeglossaries

\setglossarystyle{cbl-gloss}

%% From
%% <http://tex.stackexchange.com/questions/118939/add-watermark-that-overlays-the-images>
%% <http://tex.stackexchange.com/questions/132582/transparent-foreground-watermark>
\usepackage[printwatermark]{xwatermark}
%% Water mark in the background
%\newwatermark[allpages,color=red!10,angle=45,scale=3,xpos=0,ypos=0]{DRAFT}

\newsavebox\mydraftbox
\savebox\mydraftbox{\tikz[color=red,opacity=0.3]\node{DRAFT};}
\newwatermark*[allpages,angle=45,scale=6,xpos=-20,ypos=15]{\usebox\mydraftbox}


\title{Algorithm}
\author{Zakariyya Mughal}
\date{2015-12-12}

\begin{document}
\maketitle
\tableofcontents

For the following, all indices are 0-based in order to match the
indexing found in the \gls{orionc} code and reduce the need for
subtracting one when calculating using indices.

\section{Segmentation}

\subsection{Input}

\newcommand{\foobar}[1]{\bm{#1}}

\begin{description}
	\item[\(\InputVolumeName{}\)] is the input volume of dimensions
		\(\InputVolumeDimensions{}\) which contains the intensity values taken
		from the microscope modality.
	\item[\(\RadiiScalesName{}\)] is a \Dim{\RadiiScalesName}-tuple where each element
		\(\RadiiScalesElem \in \RadiiScalesName\)
		represents the radius of the neurite to segment.
	\item[\(\LaplacianFilterApproxDegree{}\)] is the degree of the
		exponential Taylor series used to calculate the approximation
		to the Laplacian filter. This is set to \(\num{60}\) based on Paul's
		experiments.
	\item[\(\LaplacianFilterScaleFactor{}\)] is a factor that
		is used to find the scale of the Laplacian filter
		filter for which a tubular structure with a radius
		\(\RadiiScalesElem \in \RadiiScalesName\)
		gives a maximal response. This is fixed for a
		given value of \(\LaplacianFilterApproxDegree\).
\end{description}

\subsection{Extract features}

\subsubsection{Laplacian scale parameter}

\(\LaplacianScalesName\) is a \Dim{\RadiiScalesName}-tuple such that
\begin{equation*}
	\LaplacianScalesName \ni \LaplacianScalesElem{} = \frac{\RadiiScalesElem{}}{\LaplacianFilterScaleFactor}
\end{equation*}

\todo{explain that \(\LaplacianFilterScaleFactor = \num{0.66}\) was chosen by using linear regression and fixing
\(\LaplacianFilterApproxDegree{} = \num{60}\).}

\subsubsection{Hessian scale parameter}

\(\HessianScalesName\) is a \Dim{\RadiiScalesName}-tuple where each element
\begin{equation*}
	\HessianScalesName \ni \HessianScalesElem{} = \RadiiScalesElem{}
\end{equation*}
that is, the radius at which to segment.

\subsubsection{Laplacian filter (HDAF)}

\(\LaplacianOutputVolumeName\) is a \Dim{\RadiiScalesName}-tuple
where each element \(\LaplacianOutputVolumeElem \in \LaplacianOutputVolumeName\)
is a \Dim{\InputVolumeName}-dimensional volume
that contains the filter response for the corresponding Laplacian scale \(\LaplacianScalesElem\).

\subsubsection{Hessian filter}


Hessian filter is defined for a function \(\FuncTo{f}{\Sreal^{n}}{\Sreal}\)
as
\begin{equation*}
	H(f)_{i,j}(\vect{x}) = \frac{\partial{}^{2} f(\vect{x})}{\partial{}x_{i}\partial{}x_{j}}
\end{equation*}
for a point \(\vect{x} = (x_0, x_1, x_2, \dotsc, x_{n-1}) \in \Sreal^{n}\)

For a 3D volume \(\tens{V}\), we compute the Hessian as
\begin{equation*}
	H(\tens{V})_{i,j}(\vect{x}) = \frac{\partial{}^{2} \tens{V}(\vect{x})}{\partial{}x_{i}\partial{}x_{j}}
\end{equation*}
where \(\vect{x} = (x_{0}, x_{1}, x_{2})\) is a point in the 3D
coordinate system of \(\tens{V}\).

\todo{explain how a Gaussian 2nd derivative filter is used to smooth piecewise noise from discrete volume}

\(\EigFirstName\), \(\EigSecondName\), and \(\EigThirdName\) are
are each \Dim{\RadiiScalesName}-tuples where the elements
\begin{equation*}
\begin{aligned}
	\EigFirstElem  &\in \EigFirstName  \\
	\EigSecondElem &\in \EigSecondName \\
	\EigThirdElem  &\in \EigThirdName  \\
\end{aligned}
\end{equation*}
are each \Dim{\InputVolumeName}-dimensional volumes
containing the first, second, and third eigenvalues respectively of \(H(\InputVolumeName)\)
for scale \(\HessianScalesElem\).
Each 3-tuple of eigenvalues is sorted so that
\(\abs{\EigFirstElem\InputVolumeElemIndices}
\le \abs{\EigSecondElem\InputVolumeElemIndices}
\le \abs{\EigThirdElem\InputVolumeElemIndices}\)
(i.e., the eigenvalues are sorted in ascending
order).

\subsection{Background training set}

The training set for the the background class is made up of pixels that are
either on the background

\subsubsection{Maximal Laplacian filter response}

\begin{equation*}
\MaxResponseLaplacianElemIndices = \argmax_{s} \(\LaplacianOutputVolumeElemIndices\)
\end{equation*}
where \(\MaxResponseLaplacianName\) is a volume of dimension
\(\Dim{\InputVolumeName}\) where each element is an index such that
all elements satisfy
\(0 \le \MaxResponseLaplacianElemIndices < \Dim{\RadiiScalesName}\)


\subsubsection{Feature vector}


\begin{equation*}
\FeaturesElemIndices = \Set{
\left(
	\EigSecondElemIndices,
	\EigThirdElemIndices
\right)
|
s = R_{\mathrm{max}}^{L}\InputVolumeIndices
}
\end{equation*}
where \(\FeaturesName\) is a volume of dimensions \(\Dim{\InputVolumeName}\) with
elements that are each tuples of length two, that is, the number
of eigenvalues used as features.

\subsection{Discriminant function}

\subsubsection{Input}

\begin{description}
	\item[\(\FeaturesName\)] is a volume that contains the feature vector for each
		voxel in the volume.
	\item[\(\HistogramBinCount\)] is a scalar number of bins.
\end{description}

\subsubsection{Bin bounds and edges}

The 2D histogram minima and maxima are computed for each feature
\begin{equation*}
	\HistogramMinElem = \min \Set{ f_{\FeatureIndex} | f_{\FeatureIndex} \in ( f_0, f_1 ) = f \in \FeaturesName }
\end{equation*}
\begin{equation*}
	\HistogramMaxElem = \max \Set{ f_{\FeatureIndex} | f_{\FeatureIndex} \in ( f_0, f_1 ) = f \in \FeaturesName }
\end{equation*}

Bin edges are computed for each \(\mnth{{\FeatureIndex}}\) feature
\begin{equation*}
	b_{\FeatureIndex}[w] =
		\HistogramMinElem +
		\frac{ \HistogramMaxElem  - \HistogramMinElem  }
		     { \HistogramBinCount } \, w
\end{equation*}
where \(w\) is the index of the leading edge to the
\(\mnth{w}\) bin and \( 0 \le w \le  \HistogramBinCount \).

\subsubsection{Histogram accumulation}

\begin{equation*}
	H[x,y] =  \Dim{ \BinFunc(\FeaturesName, x, y) }
\end{equation*}
where \(H[x,y]\) is a \(\HistogramBinCount \times \HistogramBinCount\) matrix and
where \(\operatorname{Bin}(\FeaturesName, x, y)\) is the set of all tuples
\((f_0, f_1) \in \FeaturesName\) such that
\begin{enumerate}
	\item \(b_0[x] \le f_0 < b_0[x+1]\) and
	\item \(b_1[y] \le f_1 < b_1[y+1]\).
\end{enumerate}



\section{Tracing}

\end{document}
