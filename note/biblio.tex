\documentclass[12pt]{article}
\usepackage[utf8]{inputenc}
\usepackage[british]{babel}

\usepackage{amsmath}
\usepackage{bm} % bold mathematics (\bm command)
\usepackage{tocbibind}

\usepackage[table,x11names,rgb]{xcolor}
\usepackage{tikz}
\usetikzlibrary{snakes,arrows,shapes}
\usetikzlibrary{positioning}

\usepackage{hyperref}
\hypersetup{
    colorlinks,
    linkcolor={red!50!black},
    citecolor={blue!50!black},
    urlcolor={blue!80!black}
}

%% glossaries needs to be after hyperref
\usepackage[toc,acronym]{glossaries}
\glsdisablehyper % only link from the glossary to the text
\glsnopostdottrue % no period at the end of a glossary entry

\newcommand{\cblglossdelimiter}{:}
\newglossarystyle{cbl-gloss}{%
\setglossarystyle{list}% base this style on the list style
\renewcommand*{\glossentry}[2]{%
\item[\glsentryitem{##1}%
  \glstarget{##1}{\glossentryname{##1}\cblglossdelimiter}]
  \glossentrydesc{##1}\glspostdescription\space ##2}
}

%% For displaying algorithms using algorithmicx
\usepackage{algorithm}
\usepackage{algpseudocode}
%% make \listofalgorithms work with tocbibind.
%% From "packages algorithm and tocbibind" <http://newsgroups.derkeiler.com/Archive/Comp/comp.text.tex/2005-10/msg01099.html>
\makeatletter
\let\l@algorithm\l@figure
\makeatother
\renewcommand{\listofalgorithms}{\begingroup
  \tocfile{List of Algorithms}{loa}
\endgroup}


\usepackage{url}
\usepackage{longtable}
\usepackage{float}
\usepackage{graphicx}
\usepackage{mathtools}
\usepackage{multirow,booktabs}
%\usepackage[authoryear,sort,comma]{natbib}
%\newcommand{\autocite}[1]{\citep{#1}}
\usepackage[doublespacing]{setspace}
\usepackage{caption} % sets the captions to singlespacing
\captionsetup[figure]{labelfont=it}

% style=authoryear
\usepackage[%
	bibstyle=ieee,citestyle=numeric-comp,%
	sorting=none,backend=biber,%
	%maxcitenames=1,%
	urldate=long,%
	isbn=false,url=false % remove extra info
] {biblatex}

\usepackage[toc,page]{appendix}

\usepackage{siunitx}
\usepackage[inline]{enumitem}
% from <http://tex.stackexchange.com/questions/56249/enumitem-package-and-description-lists>
% use with enumitem like:
%     \begin{description}[font=\textpluscolon]
%         \item[A] ...
%         \item[B] ...
%     \end{description}
\newcommand*{\textpluscolon}[1]{{#1:}}

\usepackage{attrib}
\usepackage{listings}


%\usepackage{epsfig}
%\usepackage{epsf}

%%%%%%%%%%%%%%%%%%%%%%%%%%%%%%%%%%%%%%%%%%%%%%%%%% {{{
\usepackage{usebib}
%% prints out the info for a citation key:
%%     \printarticle{Author00}
\newcommand{\printarticle}[1]{\citeauthor{#1}, ``\usebibentry{#1}{title}''}
%%%%%%%%%%%%%%%%%%%%%%%%%%%%%%%%%%%%%%%%%%%%%%%%%% }}}

%% Fancy quote %%
%% adapted from <http://tex.stackexchange.com/questions/53377/inspirational-quote-at-start-of-chapter/53452#53452>
\usepackage{quotchap}
\definecolor{quotemark}{gray}{0.7}
\newenvironment{fancyquote}%
	{%
	    \vspace{1em}%
	    \singlespacing
	    \noindent%
		 \begin{picture}(0,0)%
		 \put(-15,-0){\makebox(0,0){\scalebox{3}{\textcolor{quotemark}{``}}}}%
		 \end{picture}%
	\footnotesize\upshape%
	}%
	{%
	 \par%
	 \makebox[0pt][l]{%
	 \hspace{\linewidth}%
	 \begin{picture}(0,0)(0,0)%
	 \put(15,20){\makebox(0,0){%
	 \scalebox{3}{\color{quotemark}''}}}%
	 \end{picture}}%
	   \vspace{-2.5em}%
	}%

%% Reference description environment
%% From <http://tex.stackexchange.com/questions/1230/reference-name-of-description-list-item-in-latex>
%%
%% Usage:
%%
%%     \begin{description}
%%         \item [Vehicle\label{itm:vehicle}] Something
%%         \item [Bus\label{itm:bus}] A type of \ref{itm:vehicle}
%%         \item [Car\label{itm:car}] A type of \ref{itm:vehicle} smaller than a \ref{itm:bus}
%%     \end{description}
%%
%%     The item `\ref{itm:bus}' is listed on page~\pageref{itm:bus} in section~\nameref{itm:bus}.

\usepackage{nameref}

\makeatletter
\let\orgdescriptionlabel\descriptionlabel
\renewcommand*{\descriptionlabel}[1]{%
  \let\orglabel\label
  \let\label\@gobble
  \phantomsection
  \edef\@currentlabel{#1}%
  %\edef\@currentlabelname{#1}%
  \let\label\orglabel
  \orgdescriptionlabel{#1}%
}
\makeatother


%% number biblatex bibliography section when the biblatex style is not
%% numeric (e.g., authoryear)
%\defbibenvironment{bibliography}
  %{\enumerate
	  %\singlespacing
     %{}
     %{\setlength{\leftmargin}{\bibhang}%
      %\setlength{\itemindent}{-\leftmargin}%
      %\setlength{\itemsep}{\bibitemsep}%
      %\setlength{\parsep}{\bibparsep}}}
  %{\endenumerate}
  %{\item}

%% Command used to create description like items inside an `enumerate` environment
%%
%% e.g., with enumitem's inline enumerate* environment:
%%
%%     \begin{enumerate*}[label={\alph*)}]
%%       \enumdescitem{First} example
%%       \enumdescitem{Second} example
%%       \enumdescitem{Third} example
%%     \end{enumerate*}
\newcommand\enumdescitem[1]{\item{\bfseries#1:\,}}

\usepackage[nameinlink,capitalize]{cleveref}

%% Mathematical commands
\newcommand{\freqdom}[1]{\widehat{#1}}
\newcommand{\Convolve}{\mathop{\ast}}%
\newcommand{\HadamardProd}{\mathop{\odot}}%

\newcommand{\FourierTrans}[1]{\ensuremath{\mathcal{F}\left\{#1\right\}}}
\newcommand{\IFourierTrans}[1]{\ensuremath{\mathcal{F}^{-1}\left\{#1\right\}}}

\bibinput{thesis}
\bibliography{thesis}

\title{Annotated bibliography}
\author{Zakariyya Mughal}
\date{2015-12-12}
\begin{document}
\singlespacing
\maketitle
\tableofcontents

% TODO take these and repurpose for 01_introduction/introduction.tex
% literature review

\section{Datasets}

\begin{enumerate}[label={}]
		\enumdescitem{\printarticle{Duke-Southampton-archive:Cannon:1998}}
			This paper describes a dataset of neuron
			morphology that was developed as a
			collaboration between Duke University and
			Southampton University. This early dataset
			introduced the SWC format for representing
			neuronal arbors.

		\enumdescitem{\printarticle{Ascoli2007}}
			This paper describes the
			NeuroMorpho.Org project which collects
			neuronal morphology submitted from various
			laboratories. These submissions are used
			to perform statistical analysis in order
			to characterise different kinds of
			neurons. These data are taken from various
			cell types and different species. Each
			reconstruction has metadata that can be
			searched to narrow down to certain kinds
			of neuron sources.

		\enumdescitem{\printarticle{DIADEM-dataset:Brown:2011}}
			This paper describes the datasets from the
			DIADEM Challenge. Each dataset was
			obtained from different laboratories using
			different protocols for data acquisition.
\end{enumerate}

\section{History}

\begin{enumerate}[label={}]
	\enumdescitem{\printarticle{DIADEM&Beyond:Liu:2011}}
		This editorial paper covers details of the DIADEM
		Challenge and summarises the results of the
		algorithms. It notes that many of the algorithms
		require human intervention to achieve a high level
		of accuracy against the gold standard tracings. It
		also notes that while many algorithms worked on
		multiple datasets, none of the algorithms were
		able to generalise to \emph{all} the datasets.
		It also describes several directions for future
		research:
		\begin{enumerate}[label={}]
			\enumdescitem{Time-lapse data}
				Moving from 3D volume data to 4D
				image sequences.
			\enumdescitem{Subcellular structures}
				Using markers to identify sites on the
				neuron itself that can be identified.
			\enumdescitem{Dense trees} Working with
				data that contains dense neuronal arbors
				as opposed to sparse neuronal cells that
				were used in many of the DIADEM datasets.
			\enumdescitem{Long-distance dendritic projections}
				Expand the dataset to include data
				that has longer neuronal
				projections that are common in
				dendrites.
			\enumdescitem{Electron microscopy}
				The DIADEM datasets all came from
				optical microscopy.
				Neuroscientists also make heavy
				use of EM data and this data is
				also useful for studying neuronal
				structures.
			\enumdescitem{Moving to neural circuits and the connectome}
				The DIADEM datasets only have
				single neurons, but
				neuroscientists eventually want to
				move on to imaging multiple
				neurons to see how the structure
				of connections is important for
				signaling.
		\end{enumerate}

	\enumdescitem{\printarticle{NeuroMorphTrends:Halavi:2012}}
		This review article provides a literature review
		of the various techniques used for neuron tracing
		using computers. This includes the early history
		from the late 1960s onwards towards the DIADEM
		Challenge. In particular, the paper provides an
		overview of the brain regions and cell types that
		are studied most often and the imaging modalities
		that are used by neuromorphologists. This data is
		obtained through literature mining of 902
		publications. In addition, the authors attempted
		to obtain the data from the papers that they cover
		in their literature mining survey and the results
		of this are available on NeuroMorpho.Org.

	\enumdescitem{\printarticle{NeuroTracePerspect:Meijering:2010}}
		This review article describes the difficulties faced when
		trying to acquire neuron morphology data and the
		various computational tasks that must be
		performed in order to perform digital
		reconstructions of the neuron data. It also
		contains a survey of various reconstruction tools
		that are available. This paper occurred before the
		completion of the DIADEM Challenge, so it does not
		cover later advances in in neuron tracing.
\end{enumerate}

\section{Methods}

\begin{enumerate}[label={}]
		\enumdescitem{\printarticle{Bauer2010}}

		\enumdescitem{\printarticle{MIA-anisotropic-path-searching-Xie2011}}

		\enumdescitem{\printarticle{MICCAI-anisotropic-path-searching-Xie2010}}

		\enumdescitem{\printarticle{Jeong2015}}

		\enumdescitem{\printarticle{Luo2015}}

		\enumdescitem{\printarticle{De2015}}

		\enumdescitem{\printarticle{Gulyanon2015}}

		\enumdescitem{\printarticle{ORION_Santamaria-Pang2015}}

		\enumdescitem{\printarticle{Mukherjee2014}}

		\enumdescitem{\printarticle{Hernandez-Herrera2014}}

		\enumdescitem{\printarticle{Basu2014}}

		\enumdescitem{\printarticle{Xiao2013}}

		\enumdescitem{\printarticle{Jimenez2013}}

		\enumdescitem{\printarticle{Basu2013}}

		\enumdescitem{\printarticle{Mukherjee2013}}

		\enumdescitem{\printarticle{Hernandez-Herrera2013}}

		\enumdescitem{\printarticle{Ming2013}}

		\enumdescitem{\printarticle{Lee2012}}

		\enumdescitem{\printarticle{Czarnecki2012}}
\end{enumerate}


\section{Metrics}

\begin{enumerate}[label={}]
		\enumdescitem{\printarticle{Mayerich2011}}

		\enumdescitem{\printarticle{Mayerich2012}}

		\enumdescitem{\printarticle{btmorph-Torben-Nielsen2014}}

		\enumdescitem{\printarticle{Costa2014}}

		\enumdescitem{\printarticle{Gillette2015}}
\end{enumerate}

\end{document}
