\documentclass[12pt]{article}
\usepackage[utf8]{inputenc}
\usepackage[british]{babel}

\usepackage{amsmath}
\usepackage{bm} % bold mathematics (\bm command)
\usepackage{tocbibind}

\usepackage[table,x11names,rgb]{xcolor}
\usepackage{tikz}
\usetikzlibrary{snakes,arrows,shapes}
\usetikzlibrary{positioning}

\usepackage{hyperref}
\hypersetup{
    colorlinks,
    linkcolor={red!50!black},
    citecolor={blue!50!black},
    urlcolor={blue!80!black}
}

%% glossaries needs to be after hyperref
\usepackage[toc,acronym]{glossaries}
\glsdisablehyper % only link from the glossary to the text
\glsnopostdottrue % no period at the end of a glossary entry

\newcommand{\cblglossdelimiter}{:}
\newglossarystyle{cbl-gloss}{%
\setglossarystyle{list}% base this style on the list style
\renewcommand*{\glossentry}[2]{%
\item[\glsentryitem{##1}%
  \glstarget{##1}{\glossentryname{##1}\cblglossdelimiter}]
  \glossentrydesc{##1}\glspostdescription\space ##2}
}

%% For displaying algorithms using algorithmicx
\usepackage{algorithm}
\usepackage{algpseudocode}
%% make \listofalgorithms work with tocbibind.
%% From "packages algorithm and tocbibind" <http://newsgroups.derkeiler.com/Archive/Comp/comp.text.tex/2005-10/msg01099.html>
\makeatletter
\let\l@algorithm\l@figure
\makeatother
\renewcommand{\listofalgorithms}{\begingroup
  \tocfile{List of Algorithms}{loa}
\endgroup}


\usepackage{url}
\usepackage{longtable}
\usepackage{float}
\usepackage{graphicx}
\usepackage{mathtools}
\usepackage{multirow,booktabs}
%\usepackage[authoryear,sort,comma]{natbib}
%\newcommand{\autocite}[1]{\citep{#1}}
\usepackage[doublespacing]{setspace}
\usepackage{caption} % sets the captions to singlespacing
\captionsetup[figure]{labelfont=it}

% style=authoryear
\usepackage[%
	bibstyle=ieee,citestyle=numeric-comp,%
	sorting=none,backend=biber,%
	%maxcitenames=1,%
	urldate=long,%
	isbn=false,url=false % remove extra info
] {biblatex}

\usepackage[toc,page]{appendix}

\usepackage{siunitx}
\usepackage[inline]{enumitem}
% from <http://tex.stackexchange.com/questions/56249/enumitem-package-and-description-lists>
% use with enumitem like:
%     \begin{description}[font=\textpluscolon]
%         \item[A] ...
%         \item[B] ...
%     \end{description}
\newcommand*{\textpluscolon}[1]{{#1:}}

\usepackage{attrib}
\usepackage{listings}


%\usepackage{epsfig}
%\usepackage{epsf}

%%%%%%%%%%%%%%%%%%%%%%%%%%%%%%%%%%%%%%%%%%%%%%%%%% {{{
\usepackage{usebib}
%% prints out the info for a citation key:
%%     \printarticle{Author00}
\newcommand{\printarticle}[1]{\citeauthor{#1}, ``\usebibentry{#1}{title}''}
%%%%%%%%%%%%%%%%%%%%%%%%%%%%%%%%%%%%%%%%%%%%%%%%%% }}}

%% Fancy quote %%
%% adapted from <http://tex.stackexchange.com/questions/53377/inspirational-quote-at-start-of-chapter/53452#53452>
\usepackage{quotchap}
\definecolor{quotemark}{gray}{0.7}
\newenvironment{fancyquote}%
	{%
	    \vspace{1em}%
	    \singlespacing
	    \noindent%
		 \begin{picture}(0,0)%
		 \put(-15,-0){\makebox(0,0){\scalebox{3}{\textcolor{quotemark}{``}}}}%
		 \end{picture}%
	\footnotesize\upshape%
	}%
	{%
	 \par%
	 \makebox[0pt][l]{%
	 \hspace{\linewidth}%
	 \begin{picture}(0,0)(0,0)%
	 \put(15,20){\makebox(0,0){%
	 \scalebox{3}{\color{quotemark}''}}}%
	 \end{picture}}%
	   \vspace{-2.5em}%
	}%

%% Reference description environment
%% From <http://tex.stackexchange.com/questions/1230/reference-name-of-description-list-item-in-latex>
%%
%% Usage:
%%
%%     \begin{description}
%%         \item [Vehicle\label{itm:vehicle}] Something
%%         \item [Bus\label{itm:bus}] A type of \ref{itm:vehicle}
%%         \item [Car\label{itm:car}] A type of \ref{itm:vehicle} smaller than a \ref{itm:bus}
%%     \end{description}
%%
%%     The item `\ref{itm:bus}' is listed on page~\pageref{itm:bus} in section~\nameref{itm:bus}.

\usepackage{nameref}

\makeatletter
\let\orgdescriptionlabel\descriptionlabel
\renewcommand*{\descriptionlabel}[1]{%
  \let\orglabel\label
  \let\label\@gobble
  \phantomsection
  \edef\@currentlabel{#1}%
  %\edef\@currentlabelname{#1}%
  \let\label\orglabel
  \orgdescriptionlabel{#1}%
}
\makeatother


%% number biblatex bibliography section when the biblatex style is not
%% numeric (e.g., authoryear)
%\defbibenvironment{bibliography}
  %{\enumerate
	  %\singlespacing
     %{}
     %{\setlength{\leftmargin}{\bibhang}%
      %\setlength{\itemindent}{-\leftmargin}%
      %\setlength{\itemsep}{\bibitemsep}%
      %\setlength{\parsep}{\bibparsep}}}
  %{\endenumerate}
  %{\item}

%% Command used to create description like items inside an `enumerate` environment
%%
%% e.g., with enumitem's inline enumerate* environment:
%%
%%     \begin{enumerate*}[label={\alph*)}]
%%       \enumdescitem{First} example
%%       \enumdescitem{Second} example
%%       \enumdescitem{Third} example
%%     \end{enumerate*}
\newcommand\enumdescitem[1]{\item{\bfseries#1:\,}}

\usepackage[nameinlink,capitalize]{cleveref}

%% Mathematical commands
\newcommand{\freqdom}[1]{\widehat{#1}}
\newcommand{\Convolve}{\mathop{\ast}}%
\newcommand{\HadamardProd}{\mathop{\odot}}%

\newcommand{\FourierTrans}[1]{\ensuremath{\mathcal{F}\left\{#1\right\}}}
\newcommand{\IFourierTrans}[1]{\ensuremath{\mathcal{F}^{-1}\left\{#1\right\}}}


%% From
%% <http://tex.stackexchange.com/questions/118939/add-watermark-that-overlays-the-images>
%% <http://tex.stackexchange.com/questions/132582/transparent-foreground-watermark>
\usepackage[printwatermark]{xwatermark}
%% Water mark in the background
%\newwatermark[allpages,color=red!10,angle=45,scale=3,xpos=0,ypos=0]{DRAFT}

\newsavebox\mydraftbox
\savebox\mydraftbox{\tikz[color=red,opacity=0.3]\node{DRAFT};}
\newwatermark*[allpages,angle=45,scale=6,xpos=-20,ypos=15]{\usebox\mydraftbox}


\title{Experimental procedures and results}
\author{Zakariyya Mughal}
\date{2015-12-06}
\begin{document}
\singlespacing
\newcommand{\SegGroundTotal}{S_{\mathrm{G,T}}}
\newcommand{\SegAutomaticTotal}{S_{\mathrm{A,T}}}
\newcommand{\SegAutomaticCorrect}{S_{\mathrm{A,C}}}
\newcommand{\SegAutomaticMissing}{S_{\mathrm{A,miss}}}
\newcommand{\SegAutomaticExtra}{S_{\mathrm{A,extra}}}

\newcommand{\InputVolumeIndices}{[i,j,k]}
\newcommand{\ScaleIndex}{s}

%%% Input to the segmentation
\newcommand{\InputVolumeName}{\tens{V}}
\newcommand{\InputVolume}{\InputVolumeName}
\newcommand{\InputVolumeElem}{\InputVolume\InputVolumeIndices}
\newcommand{\InputVolumeDimensions}{\Dim{\InputVolumeName}_0 \times \Dim{\InputVolumeName}_1 \times \Dim{\InputVolumeName}_2}

%%% Used by the Laplacian filter
\newcommand{\LaplacianFilterApproxDegree}{d^{L}}

%%% Used to constrol the \LaplacianScales and \HessianScales
\newcommand{\RadiiScalesName}{\vect{r}}
\newcommand{\RadiiScales}{\vect{r}_{\ScaleIndex}}

%%% Input to Laplacian filter
\newcommand{\LaplacianScalesName}{\vect{\sigma}^{L}}
\newcommand{\LaplacianScales}{\vect{\sigma}^{L}_{\ScaleIndex}}

%%% Input to Hessian filter
\newcommand{\HessianScalesName}{\vect{\sigma}^{G}}
\newcommand{\HessianScales}{\vect{\sigma}^{G}_{\ScaleIndex}}

%%% Output of the Laplacian filter
\newcommand{\LaplacianOutputVolumeName}{\tens{V}^{L}}
\newcommand{\LaplacianOutputVolume}{\tens{V}^{L}_{\ScaleIndex}}
\newcommand{\LaplacianOutputVolumeElem}{\LaplacianOutputVolume\InputVolumeIndices}

% TODO Notation and stylistic conventions

\newacronym{SDLC}{SDLC}{Systems Development Life Cycle}

\newacronym{FFT}{FFT}{Fast-Fourier transform}
\newacronym{ORION}{ORION}{Online Reconstruction and functional Imaging Of Neurons}
\newacronym{API}{API}{Application Programming Interface}
\newacronym{ABI}{ABI}{Application Binary Interface}
\newacronym{DIADEM}{DIADEM}{Digital Reconstruction of Axonal and Dendritic Morphology}

\newacronym{EEG}{EEG}{Electroencephalography}
\newacronym{fMRI}{fMRI}{Functional Magnetic Resonance Imaging}
\newacronym{NITRC}{NITRC}{Neuroimaging Informatics Tools and Resources Clearinghouse}
\newacronym{NIF}{NIF}{Neuroscience Information Framework}

%% Symbols: symbols used in the text

\newglossaryentry{orionc}{%
	name={\ensuremath{\mathrm{{ORION}^{c}}}},%
	sort={sym: orionc},
	description={C version of \gls{orion3}}}

\newglossaryentry{orionmat}{%
	name={\ensuremath{\mathrm{{ORION}^{m}}}},%
	sort={sym: orionmat},
	description={MATLAB version of \gls{orion3}}}

\newglossaryentry{orion3}{%
	name={ORION3},%
	sort={sym: orion3},
	description={Version 3 of the \acrshort{ORION} algorithm described in~\autocite{ORION_Santamaria-Pang2015}}}

%% Notation: not used in the text

\newglossaryentry{not:Fourier}{%
	name={\ensuremath{       {\FourierTrans{\cdot}}}},%
	sort={not: fourier},
	description={Fourier transform of argument}}
\newglossaryentry{not:IFourier}{%
	name={\ensuremath{       {\IFourierTrans{\cdot}}}},%
	sort={not: fourier inverse},
	description={Inverse Fourier transform of argument}}

\newglossaryentry{not:FreqDomain}{%
	name={\ensuremath{       {\freqdom{x}}}},%
	sort={not: frequency domain},
	description={Frequency domain representation of signal $x$ such that $\freqdom{x} = \FourierTrans{x}$}}

\newglossaryentry{not:OpConvolv}{%
	name={\ensuremath{       {z = x \Convolve y}}},%
	sort={not: convolution},
	description={Convolution of $x$ and $y$}}
\newglossaryentry{not:OpMult}{%
	name={\ensuremath{       {z = x \HadamardProd y}}},%
	sort={not: product, Hadamard},
	description={Hadamard product (or pointwise/entrywise
product) of $\vect{x}$ and $\vect{y}$ such that $\vect{z}_i = \vect{x}_i \vect{y}_i$ }}



\makeglossaries

\setglossarystyle{cbl-gloss}

\maketitle
\tableofcontents

\section{Conversion}

\subsection{Input}

From the DIADEM dataset, choose a single kind of data set. In this case, we can
use the Neuromuscular Projection Fibers (NPF).

Technical note: Since the ORION code only as a reader for MetaImage format
files, use the pre-existing MetaImage files found at
\nolinkurl{smb://athena.cbl.uh.edu/neuron/Data/Dendrites/Diadem/Data\%20Set\%204/}.

\subsection{Procedure}

\begin{enumerate}
	\item Generate a maximum intensity projection of the
		volume.
	\item Visualize the ground truth (GT) tracing.
	\item Run the code on the given dataset using
		\gls{orionmat}
		and visualise the \gls{orionmat} tracing.
	\item Run the code on the given dataset using
		\gls{orionc}
		and visualise the \gls{orionc} tracing.
	\item Compute the Accuracy and Precision metrics between the pairs:
		\begin{enumerate*}
			\item \gls{orionmat} to GT
			\item \gls{orionc} to GT
			\item \gls{orionc} to \gls{orionmat}
		\end{enumerate*}
\end{enumerate}

\subsection{Qualitative results}

For each of the NPF volumes that are processed, create a maximum
intensity projection (MIP) and display the ground truth
\begin{center}
\renewcommand{\arraystretch}{1.5}% Spread rows out...
\begin{tabular}{cp{0.2\textwidth}p{0.2\textwidth}p{0.2\textwidth}p{0.2\textwidth}}
	\toprule
	\be NPF ID & \be MIP & \be Ground truth tracing & \be \gls{orionmat} tracing & \be \gls{orionc} tracing \\
	\midrule
	1      & \todofig{1/mip} & \todofig{1/gt_tracing} & \todofig{1/orionmat_tracing} & \todofig{1/orionc_tracing} \\%\hline
	2      & \todofig{2/mip} & \todofig{2/gt_tracing} & \todofig{2/orionmat_tracing} & \todofig{2/orionc_tracing} \\%\hline
	3      & \todofig{3/mip} & \todofig{3/gt_tracing} & \todofig{3/orionmat_tracing} & \todofig{3/orionc_tracing} \\
	\bottomrule
\end{tabular}
\end{center}

\subsection{Quantitative results}

\begin{center}
\renewcommand{\arraystretch}{1.5}% Spread rows out...
\begin{tabular}{cp{0.28\textwidth}p{0.28\textwidth}p{0.28\textwidth}}
	%%%%%%% header
	\toprule
	\multirow{2}{*}{\be NPF ID } & \multicolumn{3}{c}{\be Metrics} \\
				  & \multicolumn{1}{c}{\be \gls{orionmat} to GT} & \multicolumn{1}{c}{\be \gls{orionc} to GT} & \multicolumn{1}{c}{\be \gls{orionc} to \gls{orionmat}} \\
	\midrule
	%%%%%%% end of header
		1  & Accuracy: * \par Precision: * & Accuracy: * \par Precision: * & Accuracy: * \par Precision: * \\%\hline
		2  & Accuracy: * \par Precision: * & Accuracy: * \par Precision: * & Accuracy: * \par Precision: * \\%\hline
		3  & Accuracy: * \par Precision: * & Accuracy: * \par Precision: * & Accuracy: * \par Precision: * \\
	\bottomrule
\end{tabular}
\end{center}

\section{Integration}

\subsection{Input}

The input is a sample of the same NPF datasets from above.

Technical note: Vaa3D can only open a few formats such as TIFF stacks, so the
original format of DIADEM dataset should be used. These are also at
\nolinkurl{smb://athena.cbl.uh.edu/neuron/Data/Dendrites/Diadem/Data\%20Set\%204/}.

\subsection{Procedure}

\begin{enumerate}
	\item Use the Vaa3D interface to open the DIADEM TIFF stacks.
	\item Start the \gls{orionc} Vaa3D plugin.
	\item Take a screenshot of the output.
\end{enumerate}

\subsection{Results}

%\todofig{Image of Vaa3D interface with tracing}
\begin{figure}[H]
\centering
\includegraphics[width=0.8\textwidth]{gfx/vaa3d_DIADEM-NPF-3-025_3D-view}
\caption{Vaa3D: Visualization of NPF025 TIFF stack}
\end{figure}

\begin{figure}[H]
\centering
\includegraphics[width=0.8\textwidth]{gfx/vaa3d_DIADEM-NPF-3-025_APP2-tracing-3D-view}
\caption{Vaa3D: Tracing of NPF025 using Vaa3D APP2 plugin}
\end{figure}

\section{Testing and reproducibility}

\subsection{Input data}

\begin{itemize}
	\item Call graph of \gls{orionmat} (a digraph where each node is a
		function and each edge is a function call relationship where the first function
		node calls the second function node).
	\item Call graph of  \gls{orionc} with corresponding function nodes to
		those in the call graph of \gls{orionmat}.
	\item A sample of the volumes in NPF above.
\end{itemize}

\subsection{Procedure}

\begin{enumerate}
	\item Run the \gls{orionmat} code on a volume from the input data.
	\item For each call made in \gls{orionmat}, capture the
		parameters, return values, and output files of each function.
	\item For each node in the call graph of \gls{orionc}, pass the
		parameters of the corresponding \gls{orionmat} call and capture
		the return values from \gls{orionc} (note that there are no
		output files since \gls{orionc} works entirely in memory)
	\item Create a table that shows the difference in the results from each
		function call.
\end{enumerate}

\subsection{Results}

\begin{figure}
%\documentclass[tikz, border=10pt]{standalone}
\tikzset{
    vertex/.style = {
        circle,
        fill            = black,
        outer sep = 2pt,
        inner sep = 1pt,
    }
}

%\begin{document}

\newcommand{\myfun}[2] {$fun^{#1}_{#2}$}
  
\begin{tikzpicture}
  % Cases
  \definecolor{lgray}{RGB}{192,192,192}
  \definecolor{ngray}{RGB}{160,160,160}
  \definecolor{dgray}{RGB}{128,128,128}
  % Cases
  \definecolor{lblue}{RGB}{102,102,255}
  \definecolor{nblue}{RGB}{51,51,255}
  \definecolor{dblue}{RGB}{0,0,255}
% Input node
\node[draw,fill=black,text=white,rotate=90] (Input) at (2,2) {Input};  
  
% Matlab nodes
\node[draw,fill=lgray,text=white] (funM1) at ( 4,0) {\myfun{M}{1}};
\node[draw,fill=lgray,text=white] (funM2) at ( 7,0) {\myfun{M}{2}};
\node[draw,fill=lgray,text=white] (funM3) at (10,0) {\myfun{M}{3}};
\node[draw,fill=lgray,text=white] (funM4) at (13,0) {\myfun{M}{4}};
\node[draw,fill=lgray,text=white] (funM5) at (16,0) {\myfun{M}{5}};

% MC nodes
\node[draw,fill=ngray,text=white] (funMC1) at ( 4,2) {\myfun{M-C}{1}};
\node[draw,fill=ngray,text=white] (funMC2) at ( 7,2) {\myfun{M-C}{2}};
\node[draw,fill=ngray,text=white] (funMC3) at (10,2) {\myfun{M-C}{3}};
\node[draw,fill=ngray,text=white] (funMC4) at (13,2) {\myfun{M-C}{4}};
\node[draw,fill=ngray,text=white] (funMC5) at (16,2) {\myfun{M-C}{5}};

% C nodes
\node[draw,fill=dgray,text=white] (funC1) at ( 4,4) {\myfun{C}{1}};
\node[draw,fill=dgray,text=white] (funC2) at ( 7,4) {\myfun{C}{2}};
\node[draw,fill=dgray,text=white] (funC3) at (10,4) {\myfun{C}{3}};
\node[draw,fill=dgray,text=white] (funC4) at (13,4) {\myfun{C}{4}};
\node[draw,fill=dgray,text=white] (funC5) at (16,4) {\myfun{C}{5}};


% Initial data conection
\draw[->,draw=dblue] (Input) to[in=180,out=0] (funC1);
\draw[->,draw=nblue] (Input) to[in=180,out=0] (funMC1);
\draw[->,draw=lblue] (Input) to[in=180,out=0] (funM1);

% C
\draw[->,draw=dblue] (funC1) to[in=180,out=0] (funC2);
\draw[->,draw=dblue] (funC2) to[in=180,out=0] (funC3);
\draw[->,draw=dblue] (funC3) to[in=180,out=0] (funC4);
\draw[->,draw=dblue] (funC4) to[in=180,out=0] (funC5);

% MC
\draw[->,draw=nblue] (funM1) to[in=180,out=0] (funMC2);
\draw[->,draw=nblue] (funM2) to[in=180,out=0] (funMC3);
\draw[->,draw=nblue] (funM3) to[in=180,out=0] (funMC4);
\draw[->,draw=nblue] (funM4) to[in=180,out=0] (funMC5);

% M
\draw[->,draw=lblue] (funM1) to[in=180,out=0] (funM2);
\draw[->,draw=lblue] (funM2) to[in=180,out=0] (funM3);
\draw[->,draw=lblue] (funM3) to[in=180,out=0] (funM4);
\draw[->,draw=lblue] (funM4) to[in=180,out=0] (funM5);


\end{tikzpicture}
%\end{document}
\caption{Full pipeline}
\end{figure}

\end{document}
