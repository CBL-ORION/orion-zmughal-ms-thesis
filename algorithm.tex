\documentclass[12pt]{article}
\usepackage[utf8]{inputenc}
\usepackage[british]{babel}

\usepackage{amsmath}
\usepackage{bm} % bold mathematics (\bm command)
\usepackage{tocbibind}

\usepackage[table,x11names,rgb]{xcolor}
\usepackage{tikz}
\usetikzlibrary{snakes,arrows,shapes}
\usetikzlibrary{positioning}

\usepackage{hyperref}
\hypersetup{
    colorlinks,
    linkcolor={red!50!black},
    citecolor={blue!50!black},
    urlcolor={blue!80!black}
}

%% glossaries needs to be after hyperref
\usepackage[toc,acronym]{glossaries}
\glsdisablehyper % only link from the glossary to the text
\glsnopostdottrue % no period at the end of a glossary entry

\newcommand{\cblglossdelimiter}{:}
\newglossarystyle{cbl-gloss}{%
\setglossarystyle{list}% base this style on the list style
\renewcommand*{\glossentry}[2]{%
\item[\glsentryitem{##1}%
  \glstarget{##1}{\glossentryname{##1}\cblglossdelimiter}]
  \glossentrydesc{##1}\glspostdescription\space ##2}
}

%% For displaying algorithms using algorithmicx
\usepackage{algorithm}
\usepackage{algpseudocode}
%% make \listofalgorithms work with tocbibind.
%% From "packages algorithm and tocbibind" <http://newsgroups.derkeiler.com/Archive/Comp/comp.text.tex/2005-10/msg01099.html>
\makeatletter
\let\l@algorithm\l@figure
\makeatother
\renewcommand{\listofalgorithms}{\begingroup
  \tocfile{List of Algorithms}{loa}
\endgroup}


\usepackage{url}
\usepackage{longtable}
\usepackage{float}
\usepackage{graphicx}
\usepackage{mathtools}
\usepackage{multirow,booktabs}
%\usepackage[authoryear,sort,comma]{natbib}
%\newcommand{\autocite}[1]{\citep{#1}}
\usepackage[doublespacing]{setspace}
\usepackage{caption} % sets the captions to singlespacing
\captionsetup[figure]{labelfont=it}

% style=authoryear
\usepackage[%
	bibstyle=ieee,citestyle=numeric-comp,%
	sorting=none,backend=biber,%
	%maxcitenames=1,%
	urldate=long,%
	isbn=false,url=false % remove extra info
] {biblatex}

\usepackage[toc,page]{appendix}

\usepackage{siunitx}
\usepackage[inline]{enumitem}
% from <http://tex.stackexchange.com/questions/56249/enumitem-package-and-description-lists>
% use with enumitem like:
%     \begin{description}[font=\textpluscolon]
%         \item[A] ...
%         \item[B] ...
%     \end{description}
\newcommand*{\textpluscolon}[1]{{#1:}}

\usepackage{attrib}
\usepackage{listings}


%\usepackage{epsfig}
%\usepackage{epsf}

%%%%%%%%%%%%%%%%%%%%%%%%%%%%%%%%%%%%%%%%%%%%%%%%%% {{{
\usepackage{usebib}
%% prints out the info for a citation key:
%%     \printarticle{Author00}
\newcommand{\printarticle}[1]{\citeauthor{#1}, ``\usebibentry{#1}{title}''}
%%%%%%%%%%%%%%%%%%%%%%%%%%%%%%%%%%%%%%%%%%%%%%%%%% }}}

%% Fancy quote %%
%% adapted from <http://tex.stackexchange.com/questions/53377/inspirational-quote-at-start-of-chapter/53452#53452>
\usepackage{quotchap}
\definecolor{quotemark}{gray}{0.7}
\newenvironment{fancyquote}%
	{%
	    \vspace{1em}%
	    \singlespacing
	    \noindent%
		 \begin{picture}(0,0)%
		 \put(-15,-0){\makebox(0,0){\scalebox{3}{\textcolor{quotemark}{``}}}}%
		 \end{picture}%
	\footnotesize\upshape%
	}%
	{%
	 \par%
	 \makebox[0pt][l]{%
	 \hspace{\linewidth}%
	 \begin{picture}(0,0)(0,0)%
	 \put(15,20){\makebox(0,0){%
	 \scalebox{3}{\color{quotemark}''}}}%
	 \end{picture}}%
	   \vspace{-2.5em}%
	}%

%% Reference description environment
%% From <http://tex.stackexchange.com/questions/1230/reference-name-of-description-list-item-in-latex>
%%
%% Usage:
%%
%%     \begin{description}
%%         \item [Vehicle\label{itm:vehicle}] Something
%%         \item [Bus\label{itm:bus}] A type of \ref{itm:vehicle}
%%         \item [Car\label{itm:car}] A type of \ref{itm:vehicle} smaller than a \ref{itm:bus}
%%     \end{description}
%%
%%     The item `\ref{itm:bus}' is listed on page~\pageref{itm:bus} in section~\nameref{itm:bus}.

\usepackage{nameref}

\makeatletter
\let\orgdescriptionlabel\descriptionlabel
\renewcommand*{\descriptionlabel}[1]{%
  \let\orglabel\label
  \let\label\@gobble
  \phantomsection
  \edef\@currentlabel{#1}%
  %\edef\@currentlabelname{#1}%
  \let\label\orglabel
  \orgdescriptionlabel{#1}%
}
\makeatother


%% number biblatex bibliography section when the biblatex style is not
%% numeric (e.g., authoryear)
%\defbibenvironment{bibliography}
  %{\enumerate
	  %\singlespacing
     %{}
     %{\setlength{\leftmargin}{\bibhang}%
      %\setlength{\itemindent}{-\leftmargin}%
      %\setlength{\itemsep}{\bibitemsep}%
      %\setlength{\parsep}{\bibparsep}}}
  %{\endenumerate}
  %{\item}

%% Command used to create description like items inside an `enumerate` environment
%%
%% e.g., with enumitem's inline enumerate* environment:
%%
%%     \begin{enumerate*}[label={\alph*)}]
%%       \enumdescitem{First} example
%%       \enumdescitem{Second} example
%%       \enumdescitem{Third} example
%%     \end{enumerate*}
\newcommand\enumdescitem[1]{\item{\bfseries#1:\,}}

\usepackage[nameinlink,capitalize]{cleveref}

%% Mathematical commands
\newcommand{\freqdom}[1]{\widehat{#1}}
\newcommand{\Convolve}{\mathop{\ast}}%
\newcommand{\HadamardProd}{\mathop{\odot}}%

\newcommand{\FourierTrans}[1]{\ensuremath{\mathcal{F}\left\{#1\right\}}}
\newcommand{\IFourierTrans}[1]{\ensuremath{\mathcal{F}^{-1}\left\{#1\right\}}}


%% From
%% <http://tex.stackexchange.com/questions/118939/add-watermark-that-overlays-the-images>
%% <http://tex.stackexchange.com/questions/132582/transparent-foreground-watermark>
\usepackage[printwatermark]{xwatermark}
%% Water mark in the background
%\newwatermark[allpages,color=red!10,angle=45,scale=3,xpos=0,ypos=0]{DRAFT}

\newsavebox\mydraftbox
\savebox\mydraftbox{\tikz[color=red,opacity=0.3]\node{DRAFT};}
\newwatermark*[allpages,angle=45,scale=6,xpos=-20,ypos=15]{\usebox\mydraftbox}

\usepackage{braket}
\usepackage{breqn}

\title{Algorithm}
\author{Zakariyya Mughal}
\date{2015-12-12}

\newcommand{\Dim}[1]{\ensuremath{\left|#1\right|}}
\newcommand{\InputVolumeIndices}{\ensuremath{[i,j,k]}}
\newcommand{\InputVolumeName}{\ensuremath{V}}
\newcommand{\InputVolume}{\ensuremath{\InputVolumeName\InputVolumeIndices}}
\newcommand{\InputVolumeDimensions}{\ensuremath{\Dim{\InputVolumeName}_0 \times \Dim{\InputVolumeName}_1 \times \Dim{\InputVolumeName}_2}}

\newcommand{\LaplacianFilterApproxDegree}{\ensuremath{d^{L}}}
\newcommand{\RadiiScalesName}{\ensuremath{r}}
\newcommand{\RadiiScales}{\ensuremath{\RadiiScalesName_{s}}}

\begin{document}
\maketitle
\tableofcontents

For the following, all indices are 0-based in order to match the
indexing found in the \gls{orionc} code and reduce the need for
subtracting one when calculating using indices.

\section{Segmentation}

\subsection{Input}

\begin{description}
	\item[\InputVolume{}] is the input volume of dimensions
		\InputVolumeDimensions{} which contains the intensity values taken
		from the microscope modality.
	\item[\RadiiScales{}] is a \Dim{\RadiiScalesName}-vector where each element
		represents the radius of the neurite to segment.
	\item[\LaplacianFilterApproxDegree{}] is the degree of the
		Taylor series used to calculate the approximation
		to the Laplacian filter.
\end{description}

\subsection{Extract features}

\paragraph{Laplacian scale parameter}

\(\sigma^{L}_{s}\) is a \Dim{\RadiiScalesName}-vector

\paragraph{Hessian scale parameter}

\(\sigma^{G}_{s}\) is a \Dim{\RadiiScalesName}-vector

\paragraph{Laplacian filter (HDAF)}

\(V^{L}_{s}\InputVolumeIndices\) is a \Dim{\RadiiScalesName}-vector of
\Dim{\InputVolumeName} dimensional volumes where each element
contains the filter response for scale \(\sigma^{L}_{s}\).

\paragraph{Hessian filter}

\(\lambda_{s,1}\InputVolumeIndices\),
\(\lambda_{s,2}\InputVolumeIndices\), and
\(\lambda_{s,3}\InputVolumeIndices\) are
a collection of \Dim{\RadiiScalesName}-count
\Dim{\InputVolumeName} dimensional volumes where each element is
the respective eigenvalues \(\lambda_1 \le \lambda_2 \le \lambda_3\) for scale
\(\sigma^{G}_{s}\) (i.e., the eigenvalues are sorted in ascending
order).

\paragraph{Maximal Laplacian filter response}

\begin{dmath*}
R_{\mathrm{max}}^{L}\InputVolumeIndices = \argmax_{s} \(V^{L}_{s}\InputVolumeIndices\)
\end{dmath*}
where \(R_{\mathrm{max}}^{L}\) is a volume of dimension
\(\Dim{\InputVolumeName}\) where each element is an index such that
all elements satisfy
\(0 \le R_{\mathrm{max}}^{L}\InputVolumeIndices < \Dim{\RadiiScalesName}\)

\paragraph{Feature vector}


\begin{dmath*}
F^{G}\InputVolumeIndices = \Set{
\left(
\lambda_{r,2}\InputVolumeIndices,
\lambda_{r,3}\InputVolumeIndices
\right)
|
r = R_{\mathrm{max}}^{L}\InputVolumeIndices
}
\end{dmath*}
where \(F^{G}\) is a volume of dimensions \Dim{\InputVolume} with
elements that are each tuples of length two, that is, the number
of eigenvalues used as features.

\subsection{Discriminant function}

\subsection{Input}

\begin{description}
	\item[\(F^{G}\)] is a volume that contains the feature vector for each
		voxel in the volume.
	\item[\(H_n\)] is a scalar number of bins.
\end{description}

\subsection{Bin bounds and edges}

The 2D histogram minima and maxima are computed for each feature
\begin{dmath*}
	h_{\mathrm{min},q} = \min \Set{ f_q | f_q \in ( f_0, f_1 ) = f \in F^{G} }
\end{dmath*}
\begin{dmath*}
	h_{\mathrm{max},q} = \max \Set{ f_q | f_q \in ( f_0, f_1 ) = f \in F^{G} }
\end{dmath*}

Bin edges are computed for each \(\mnth{q}\) feature
\begin{dmath*}
	b_q[w] =
		h_{\mathrm{min},q} +
		\frac{ h_{\mathrm{max},q}  - h_{\mathrm{min},q}  }
		     { H_n } \, w
\end{dmath*}
where \(w\) is the index of the leading edge to the
\(\mnth{w}\) bin and \( 0 \le w \le  H_n \).

\subsection{Histogram accumulation}

\begin{dmath*}
	H[x,y] =  \Dim{ \operatorname{Bin}(F^G, x, y) }
\end{dmath*}
where \(H[x,y]\) is a \(H_n \times H_n\) matrix and
where \(\operatorname{Bin}(F^G, x, y)\) is the set of all tuples
\((f_0, f_1) \in F^G\) such that
\begin{enumerate}
	\item \(b_0[x] \le f_0 < b_0[x+1]\) and
	\item \(b_1[y] \le f_1 < b_1[y+1]\).
\end{enumerate}



\section{Tracing}

\end{document}
