
\begin{savequote}[0.55\linewidth]
	\begin{fancyquote}
	One of my most productive days was throwing away 1000 lines of code.
	\end{fancyquote}
	\qauthor{Ken Thompson}
	\begin{fancyquote}
		Programming, programming, all through the night,\\
		We're stuck here until our new program works right.\\
		Programming, programming, isn't it fun?\\
		The maintainance starts when debugging is done!
	\end{fancyquote}
	\qauthor{Steve Savitzky in \emph{The Programmer's Alphabet}, 1981~\autocite{ProgrammersAlphabet}}
\end{savequote}
\chapter{Implementation}\label{ch:implementation}
% TODO
% 4. Implementation
%    4.1 Data structures
%        [ n-d array (e.g. row-major, column-major) ]
%    4.2 Prototyping components
%    4.3 Integration
%        [ Vaa3D & BigNeuron ]
%    4.4 Build system

As discussed in \cref{ch:design}, the Implementation stage
starts by converting the MATLAB code to native code by following
the 

The Implementation
Using the Analysis and Design from the previous chapters

\section{Data structures}\label{subsec:impl:ds}

% column-major, row-major

The main data structure used in the implementation is an
n-dimensional array or tensor. Since the algorithm is meant to
work with 3D data, the 

MATLAB also has a similar data structure that can be accessed via the MEX interface.

\section{Prototyping components}

\section{Integration}

