
\begin{savequote}[0.55\linewidth]
	\begin{fancyquote}
	One of my most productive days was throwing away 1000 lines of code.
	\end{fancyquote}
	\qauthor{Ken Thompson}
	\begin{fancyquote}
		Programming, programming, all through the night,\\
		We're stuck here until our new program works right.\\
		Programming, programming, isn't it fun?\\
		The maintenance starts when debugging is done!
	\end{fancyquote}
	\qauthor{Steve Savitzky in \emph{The Programmer's Alphabet}, 1981~\autocite{ProgrammersAlphabet}}
\end{savequote}
\chapter{Implementation}\label{ch:implementation}
% 4. Implementation
%    4.1 Data structures
%        [ n-d array (e.g. row-major, column-major) ]
%    4.2 Prototyping components
%    4.3 Integration
%        [ Vaa3D & BigNeuron ]

As discussed in \cref{ch:design}, the Implementation stage starts by converting
the MATLAB code to native code by following the same architecture as the
\gls{orionmat} code as initial point. This chapter discusses the real world
characteristics of the code used to implement \gls{orionc}.
This starts with the data structures that are used for calculations.
Implementing a data structure that captures the domain of the problem correctly
is the most important first step in any software system because once a data
structure is chosen, it can be difficult to change as every part of the code
relies on the properties of that data structure.

\section{Data structures}\label{subsec:impl:ds}

% column-major, row-major

The main data structure used in the implementation is an
\(n\)-dimensional array or tensor. Since the algorithm is meant to
work with 3D data, the \(n\)-dimensional array has a fixed \(n = 3\).
To define this data structure, we need to examine the memory layout.
There are two approaches to this: column-major and row-major. The difference between the two is based on which dimension index
changes the faster when accessing memory linearly. This difference is
illustrated in \cref{fig:col-row-major} which depicts a data that is stored in memory
with increasing value from \num{1} to \num{9}. The indices to access the same
value differ between MATLAB and C not only due to MATLAB's \num{1}-based indexing and
C's \num{0}-based indexing, but also the memory layout.
\begin{figure}
	\centering
	\begin{tabular}{cc}
		MATLAB & C\\[1ex]
		x(r,c) & x[r][c]\\[1ex]
		\begin{tabular}{cc|c|c|c|}
		\backslashbox{r}{c}
		  & \cellcolor{langM} 1 & \cellcolor{langM} 2 & \cellcolor{langM} 3 \\
		\cellcolor{langM} 1 & \cellcolor{data} 1 & \cellcolor{data} 4 & \cellcolor{data} 7 \\\hline
		\cellcolor{langM} 2 & \cellcolor{data} 2 & \cellcolor{data} 5 & \cellcolor{data} 8 \\\hline
		\cellcolor{langM} 3 & \cellcolor{data} 3 & \cellcolor{data} 6 & \cellcolor{data} 9 \\\hline
		\end{tabular}
	&
		\begin{tabular}{cc|c|c|c|}
		\backslashbox{r}{c}
				    & \cellcolor{langC} 0 & \cellcolor{langC} 1 & \cellcolor{langC} 2 \\
		\cellcolor{langC} 0 & \cellcolor{data} 1 & \cellcolor{data} 2 & \cellcolor{data} 3 \\\hline
		\cellcolor{langC} 1 & \cellcolor{data} 4 & \cellcolor{data} 5 & \cellcolor{data} 6 \\\hline
		\cellcolor{langC} 2 & \cellcolor{data} 7 & \cellcolor{data} 8 & \cellcolor{data} 9 \\\hline
		\end{tabular}
	\end{tabular}
	\caption{Difference in memory layout between MATLAB (column-major) and C (row-major).}\label{fig:col-row-major}
\end{figure}

For \gls{orionc}, the \(n\)-dimensional array  is stored as row-major
as this is the default that most C programmers expect. This array is
represented by the C structure
\begin{lstlisting}
typedef struct {
 pixel_type* p;                   /* allocated memory for data */

 size_t sz[PIXEL_NDIMS];          /* size of each dimension    */
 pixel_type spacing[PIXEL_NDIMS]; /* physical spacing          */
} ndarray3;
\end{lstlisting}
where
\computertext{pixel_type} is represents the floating point storage type used
for the \(n\)-dimensional data, \computertext{p} is a pointer to the block of
memory where the data is stored, and \computertext{PIXEL_NDIMS} is the constant
\num{3}.
 The \computertext{spacing} field is used to
store the physical spacing between items in the \(n\)-dimensional grid and is
used for normalizing calculations.

There are also other data structures that are mainly used for book-keeping and
storage of the algorithm parameters. For example, to obtain the output of the
Hessian filter, a collection of the eigenvalue features and the
associated scales used for that filter are returned by using the C structure
\begin{lstlisting}
typedef struct {
 float scale;               /* scale used for this filter result    */
 array_ndarray3* eig_feat;  /* the three eigenvalues for each voxel */
} orion_eig_feat_result;
\end{lstlisting}
This allows for access to the filter response as well as
the original parameters that generated that response.

\section{Numerical considerations}

Scientific computing often uses floating point for calculations, but the
techniques for reducing numerical error are often overlooked~\autocite{Goldberg:1991:floating}.
For example, a simple summation of a list of floating point numbers accumulates
numerical errors as more numbers are numbers are added. To compensate for this
numerical error, algorithms such as Kahan summation are used to accumulate this error
in another \emph{running compensation} variable~\autocite{Kahan:1965:summation}.

Another numerical consideration is making sure that the result of the floating
point operations can fit within the limits of the storage type. For example,
multiplying large floating point numbers can cause the result to become larger
than the largest \computertext{double} (64-bit) floating point value which
means that the result is no longer representable. This can happen when
calculating factorials for a Taylor series. To avoid this, it is important to
calculate the bounds of a calculation and provide error checking for when the
input could lead to invalid results. For a factorial function that works with
64-bit floating point data, we can calculate the bounds of input to the
function \(n!\) as \(n \in [ 0, 170 ]\). In addition, since the domain of the
input variable is so small, storing the pre-calculated data in a lookup table
can help avoid numerical errors that accumulate through multiplication.

% TODO real data -> conjugate symmetric --- r2c, c2r

\section{Prototyping components}

As mentioned before, the design of \gls{orionc} initially follows the
architecture of the \gls{orionmat} code. To accomplish this, each MATLAB
function has a corresponding C function that is located in a similarly named
file. Using the call graphs obtained in \cref{appx:matlab-call-graph}, we first
choose a function that is a leaf node in the call graph. This function requires
very little data for its inputs, so testing the \gls{orionc} implementation
against the \gls{orionmat} implementation does not require a lot of set up for
each of the parameters. In order to prototype the \gls{orionc} function,
we take a set of test parameters and apply them to the \gls{orionmat} function
and capture its output. This output is then placed inside a test file and used
to compare the C output to the MATLAB output. This procedure is then repeated
for each parent function in the call graph.

\section{Library integration}

Some of the calculations needed for this implementation require outside
libraries. In order to integrate with these libraries, it is necessary to
convert the \gls{orionc} data structures into appropriate data structures for each library.

To compute the vessellness filter, the ITK library is
used~\autocite{Ibanez:2012:ITK}. This library is widely used in biomedical
image analysis and has many implementations of filters that are commonly used
for image segmentation. The ITK library uses a data structure called
\computertext{itk::Image} for to store the inputs and outputs for calculations.
This data structure is internally very similar to the \gls{orionc}
\computertext{ndarray3} data structure as they both use a row-major block of
memory to store their data. As such, it is possible to convert the
\computertext{ndarray3} data to \computertext{itk::Image} data without having
to copy the data by sharing the buffer between the two libraries.

For computation of the fast-Fourier transform, a small library called Kiss FFT
is used~\autocite{kissfft}. This library has functions for computing an
\(n\)-dimensional \gls{FFT}. This library also uses a \(n\)-dimensional array data
structure that is similar to \computertext{ndarray3} which makes it easy to
write a wrapper to the \gls{FFT} code that takes an \computertext{ndarray3} input.
