\begin{savequote}[0.55\linewidth]
	\begin{fancyquote}
		The purpose of computing is insight, not numbers.
	\end{fancyquote}
	\qauthor{Richard W. Hamming in \emph{Numerical Methods for Scientists and Engineers}, 1962}
	\begin{fancyquote}
		There's no sense in being precise when you don't even know what
		you're talking about.
	\end{fancyquote}
	\qauthor{John von Neumann}
\end{savequote}
\chapter{Conclusion}\label{ch:conclusion}
% 6. Conclusion
%    [ Discussion of overall project results and future work. ]

The main contribution of this thesis is a neuron-morphology
reconstruction system with an automated test suite. The existence
of an automated test suite allows for future refactoring and
additions to the code with confidence that each component behaves
as expected. The consequences of this contribution go beyond just
the testing of the current state of the system --- testing allows
the code to evolve to incorporate new designs.

While the code currently relies heavily on the original
\gls{orionmat} design, removing much of the MATLAB specific design
decisions such as caching make it easier to refactor because the
only input data to a \gls{orionc} function are those which are passed in
directly as input parameters. This also leaves room for making
parts of the system run in parallel as there will be no extra
overhead or file system contention involved in reading and writing
to the disk.

Furthermore, although the main backbone of the \gls{orionc}
pipeline uses the same design as \gls{orionmat}, the
implementation inside each component calls out to generic,
reusable code. This reusable code (everything outside of the
\computertext{lib/kitchen-sink} directory) is designed so that it
can more easily be tested than the code based on \gls{orionmat}.

% Can be packaged for Vaa3D, NeuroDebian
Future work on this software system can also include packaging the
software for easy installation in repositories such as
NeuroDebian~\autocite{NeuroDebian:Halchenko:2012}. This will allow
end-users an easy way to install the software without having to
deal with building it themselves. This also serves as another way
of testing the software because these end-users will likely use
the software on a wide variety of data as well as under different
environments. This opens the door to improvements both to the
robustness of the reconstruction algorithm and the portability of
the software system.
