\begin{savequote}[0.55\linewidth]
	\begin{fancyquote}
		The trouble with computers is you \emph{play} with them. They are so
		wonderful. You have these switches --- if it's an even number you do
		this, if it's an odd number you do that --- and pretty soon you can
		do more and more elaborate things if you are clever enough, on one
		machine.
	\end{fancyquote}
	\qauthor{ Richard Feynman in \emph{Surely You're Joking, Mr. Feynman!: Adventures of a Curious Character}, 1985 }
\end{savequote}
\chapter{Design}\label{ch:design}
% 3. Design
%    [ Describes the architectural design, patterns, design
%      decisions. Describe the libraries that are used and
%      why. Documentation system. ]

% explain the importance of the Design phase
Once the planning is done, the actual technical details of the
project are determined in the Design phase. This phase is not a
discrete step that is separate from the following Implementation
phase; as the implementation continues, the Design is updated to
take into account new information. As such, it is important that
the Design is able to incorporate incremental changes otherwise
adding new changes will become difficult --- especially
when fundamental data structures need to be modified. The following
details these design decisions.

\section{Incorporation of \glsentrytext{orionmat}}

% start with MATLAB code and why
Since the algorithm already exists as a design in the \gls{orionmat}
implementation, this can be leveraged as a starting point for the
conversion. This means that the structure of \gls{orionc} will
start off with the same structure as the MATLAB version. This is
to reduce the cognitive load when rewriting and testing each
component because the inputs and outputs remain the same and
keeping the names the same makes it easier to navigate between
corresponding functions.

% describe naming convention and the way of linking back to
% original implementation
To facilitate this way of working, the filenames and directory
structure of the original MATLAB code are copied verbatim: instead
of using the \computertext{.m} extension for MATLAB files, the
\computertext{.c} extension is used. Inside each of these
\computertext{.c} files, a function signature is defined that
matches the one found in MATLAB with the exception that the name
is prefixed with \computertext{orion\_} such that a \gls{orionmat}
function named \computertext{hdaf} would appear as
\gls{orionc} function named \computertext{orion\_hdaf}.
This prefixing is common in C libraries as a way to provide
an application-specific namespace for symbols. This helps avoid
naming collisions where multiple libraries may define the same
symbol and these multiple definitions will need to be disambiguated.

\section{Algorithms and architecture}

The architecture of a software system includes both the individual
components and how they interact with one another and the core
data structures that are used to transfer data between the
components. The following description of the architecture will
approach these details from the top-down.

The ORION algorithm consists of three parts (as shown in
\cref{fig:high-level-arch} where data is drawn as dashed-line
ovals and processing is drawn as solid-lined rectangles),
namely
\begin{description}[font=\textpluscolon]
	\item[Segmentation] to label the
		foreground and the background of the image;
	\item[Registration] for aligning subvolumes so
		that they can be used to create a single volume;
		and
	\item[Tracing] to extract a centerline from the volume to
		capture the underlying neuron morphology.
\end{description}

%For the following description of the algorithm, all indices are
%0-based in order to match the indexing found in the \gls{orionc}
%code and reduce the need for subtracting one when calculating
%using indices.

%\begin{figure}
%\centering
%\resizebox{1.0\textwidth}{!}{
\begin{tikzpicture}[node distance=0.3cm]

	\tikzstyle{process} = [draw,shape=rectangle,scale=0.3]
	\tikzstyle{data} = [draw,shape=circle,style=dashed,scale=0.3]
	%
	\node (data_vol)        at (0,0)              [data] {$\mathrm{3D~volume}$};

	\node (proc_seg)        [right=of data_vol] [process] {$\mathrm{Segmentation}$};
	\node (proc_reg)        [right=of proc_seg] [process] {$\mathrm{Registration}$};
	\node (proc_trace)      [right=of proc_reg] [process] {$\mathrm{Tracing}$};

	\node (data_centerline) [right=of proc_trace] [data] {$\mathrm{Centerline}$};

	\draw [->] (data_vol)   --  (proc_seg);
	\draw [->] (proc_seg)   --  (proc_reg);
	\draw [->] (proc_reg)   --  (proc_trace);
	\draw [->] (proc_trace) --  (data_centreline);
%
\end{tikzpicture}

}
%\caption[High level diagram of ORION algorithm]{\textbf{High level
%diagram of ORION algorithm}: This high-level diagram shows that
%there are three steps to the \gls{orionmat} algorithm as described
%in the text. The input to the algorithm is a 3D volume of
%microscopy data and the output is a graph-based representation of
%the neuron morphology based on the centerline.}\label{fig:high-level-arch}
%\end{figure}

%% explain the current ORION algorithm
%\subsection{Segmentation}

%As depicted in \cref{fig:proc-segmentation}, the
%segmentation process performs one-class classification using the
%background voxels to learn a discriminant function for extracting
%vessel-like structures. The segmentation uses a multiscale approach so that
%both thin and thick neurite structures can be extracted at once from a single
%volume.

%\begin{figure}
%\centering
%\resizebox{1.0\textwidth}{!}{\begin{tikzpicture}[node distance=0.5cm]
	\tikzstyle{step} = [draw,shape=circle,scale=0.4];
	%
	\node (step_bg_train)        at (0,0)             [draw,step] {\parbox{3cm}{\centering{}Obtain background training data}};

	\node (step_feat)        [right=of step_bg_train] [draw,step] {\parbox{2cm}{\centering{}Obtain vessellness features}};
	\node (step_discrim)     [right=of step_feat]     [draw,step] {\parbox{2cm}{\centering{}Learn discriminant function}};
	\node (step_thresh)      [below=of step_discrim]  [draw,step] {\parbox{2cm}{\centering{}Threshold}};
	\node (step_post)        [left=of step_thresh]    [draw,step] {\parbox{2cm}{\centering{}Post-processing}};

	\draw [->] (step_bg_train) -- (step_feat);
	\draw [->] (step_feat) -- (step_discrim);
	\draw [->] (step_discrim) -- (step_thresh);
	\draw [->] (step_thresh) -- (step_post);
%
\end{tikzpicture}
}
%%\captionsetup{singlelinecheck=off}
%\caption[Diagram of segmentation process]{\textbf{Diagram of segmentation process}:
%\begin{enumerate*}[label={\alph*)}]
	%\enumdescitem{Obtain background training data} This step is used
		%to bootstrap the learning process so that a quick
		%approximation of the distribution of background
		%and foreground voxels can be determined
		%(\cref{alg:seg_bg_train}).
	%\enumdescitem{Obtain vessellness features} This step is used to
		%obtain features that are used to discriminate
		%vessel-like structures
		%(\cref{alg:seg_vessellness}).
	%\enumdescitem{Learn discriminant function} Here an accumulator is used to
		%characterize the distribution of the vessellness features
		%(\cref{alg:seg_discrim}).
	%\enumdescitem{Threshold} Using the discriminant function from the previous
		%step, the volume is segmented into foreground and background classes
		%(\cref{alg:seg_threshold}).
	%\enumdescitem{Post-processing} Removal of artifacts such as small connected components
		%(\cref{alg:seg_post}).
%\end{enumerate*}
%}\label{fig:proc-segmentation}
%\end{figure}

%\begin{algorithm}
	%\caption{Input parameters to segmentation}\label{alg:seg_input}
	%\begin{algorithmic}[1]
		%\State $I \gets \text{the input volume}$
		%\State $k \gets \text{the number of scales}$
		%\State $r \gets  \{ r_0, r_1, \ldots, r_{k-2}, r_{k-1} \}$
	%\end{algorithmic}
%\end{algorithm}

%The following parameters are used as the input to the segmentation
%algorithm:
%\begin{description}
	%\item[\(\InputVolumeName{}\)] is the input volume of dimensions
		%\(\InputVolumeDimensions{}\) which contains the intensity values taken
		%from the microscope modality.
	%\item[\(\RadiiScalesName{}\)] is a \Dim{\RadiiScalesName}-tuple where each element
		%\(\RadiiScalesElem \in \RadiiScalesName\)
		%represents the radius of the neurite to segment.
	%\item[\(\LaplacianFilterApproxDegree{}\)] is the degree of the
		%exponential Taylor series used to calculate the approximation
		%to the Laplacian filter.
	%\item[\(\LaplacianFilterScaleFactor{}\)] is a factor that
		%is used to find the scale of the Laplacian filter
		%filter for which a tubular structure with a radius
		%\(\RadiiScalesElem \in \RadiiScalesName\)
		%gives a maximal response. This is fixed for a
		%given value of \(\LaplacianFilterApproxDegree\).
%\end{description}

%To obtain the background training set, we use a Laplacian filter
%with a scale parameter in order to detect the edges



%\subsubsection{Laplacian scale parameter}

%\(\LaplacianScalesName\) is a \Dim{\RadiiScalesName}-tuple such that
%\begin{equation*}
	%\LaplacianScalesName \ni \LaplacianScalesElem{} = \frac{\RadiiScalesElem{}}{\LaplacianFilterScaleFactor}
%\end{equation*}

%\todo{explain that \(\LaplacianFilterScaleFactor = \num{0.66}\) was chosen by using linear regression and fixing
%\(\LaplacianFilterApproxDegree{} = \num{60}\).}

%\subsubsection{Hessian scale parameter}

%\(\HessianScalesName\) is a \Dim{\RadiiScalesName}-tuple where each element
%\begin{equation*}
	%\HessianScalesName \ni \HessianScalesElem{} = \RadiiScalesElem{}
%\end{equation*}
%that is, the radius at which to segment.

%\subsubsection{Laplacian filter (HDAF)}

%\(\LaplacianOutputVolumeName\) is a \Dim{\RadiiScalesName}-tuple
%where each element \(\LaplacianOutputVolumeElem \in \LaplacianOutputVolumeName\)
%is a \Dim{\InputVolumeName}-dimensional volume
%that contains the filter response for the corresponding Laplacian scale \(\LaplacianScalesElem\).

%\begin{algorithm}
	%\caption{Calculate}\label{alg:seg_bg_train}
	%\begin{algorithmic}[1]
		%\State $\freqdom{L}[i] \gets \Call{LaplacianFilter}{n, \sigma_{L}[i]}$
			%\Comment{Construct Laplacian filters in the frequency domain}
		%\State ${I}_{L}[i] \gets \IFourierTrans{ \freqdom{I} \HadamardProd \freqdom{L}[i] }$
		%\State
	%\end{algorithmic}
%\end{algorithm}


%\subsubsection{Hessian filter}


%Hessian filter is defined for a function \(\FuncTo{f}{\Sreal^{n}}{\Sreal}\)
%as
%\begin{equation*}
	%H(f)_{i,j}(\vect{x}) = \frac{\partial{}^{2} f(\vect{x})}{\partial{}x_{i}\partial{}x_{j}}
%\end{equation*}
%for a point \(\vect{x} = (x_0, x_1, x_2, \dotsc, x_{n-1}) \in \Sreal^{n}\)

%For a 3D volume \(\tens{V}\), we compute the Hessian as
%\begin{equation*}
	%H(\tens{V})_{i,j}(\vect{x}) = \frac{\partial{}^{2} \tens{V}(\vect{x})}{\partial{}x_{i}\partial{}x_{j}}
%\end{equation*}
%where \(\vect{x} = (x_{0}, x_{1}, x_{2})\) is a point in the 3D
%coordinate system of \(\tens{V}\).

%\todo{explain how a Gaussian 2nd derivative filter is used to smooth piecewise noise from discrete volume}

%\(\EigFirstName\), \(\EigSecondName\), and \(\EigThirdName\) are
%are each \Dim{\RadiiScalesName}-tuples where the elements
%\begin{equation*}
%\begin{aligned}
	%\EigFirstElem  &\in \EigFirstName  \\
	%\EigSecondElem &\in \EigSecondName \\
	%\EigThirdElem  &\in \EigThirdName  \\
%\end{aligned}
%\end{equation*}
%are each \Dim{\InputVolumeName}-dimensional volumes
%containing the first, second, and third eigenvalues respectively of \(H(\InputVolumeName)\)
%for scale \(\HessianScalesElem\).
%Each 3-tuple of eigenvalues is sorted so that
%\(\abs{\EigFirstElem\InputVolumeElemIndices}
%\le \abs{\EigSecondElem\InputVolumeElemIndices}
%\le \abs{\EigThirdElem\InputVolumeElemIndices}\)
%(i.e., the eigenvalues are sorted in ascending
%order).

%\subsection{Background training set}

%The training set for the the background class is made up of pixels that are
%either on the background

%\subsubsection{Maximal Laplacian filter response}

%\begin{equation*}
%\MaxResponseLaplacianElemIndices = \argmax_{s} \(\LaplacianOutputVolumeElemIndices\)
%\end{equation*}
%where \(\MaxResponseLaplacianName\) is a volume of dimension
%\(\Dim{\InputVolumeName}\) where each element is an index such that
%all elements satisfy
%\(0 \le \MaxResponseLaplacianElemIndices < \Dim{\RadiiScalesName}\)


%\subsubsection{Feature vector}


%\begin{equation*}
%\FeaturesElemIndices = \Set{
%\left(
	%\EigSecondElemIndices,
	%\EigThirdElemIndices
%\right)
%|
%s = R_{\mathrm{max}}^{L}\InputVolumeIndices
%}
%\end{equation*}
%where \(\FeaturesName\) is a volume of dimensions \(\Dim{\InputVolumeName}\) with
%elements that are each tuples of length two, that is, the number
%of eigenvalues used as features.

%\subsection{Discriminant function}

%\subsubsection{Input}

%\begin{description}
	%\item[\(\FeaturesName\)] is a volume that contains the feature vector for each
		%voxel in the volume.
	%\item[\(\HistogramBinCount\)] is a scalar number of bins.
%\end{description}

%\subsubsection{Bin bounds and edges}

%The 2D histogram minima and maxima are computed for each feature
%\begin{equation*}
	%\HistogramMinElem = \min \Set{ f_{\FeatureIndex} | f_{\FeatureIndex} \in ( f_0, f_1 ) = f \in \FeaturesName }
%\end{equation*}
%\begin{equation*}
	%\HistogramMaxElem = \max \Set{ f_{\FeatureIndex} | f_{\FeatureIndex} \in ( f_0, f_1 ) = f \in \FeaturesName }
%\end{equation*}

%Bin edges are computed for each \(\mnth{{\FeatureIndex}}\) feature
%\begin{equation*}
	%b_{\FeatureIndex}[w] =
		%\HistogramMinElem +
		%\frac{ \HistogramMaxElem  - \HistogramMinElem  }
		     %{ \HistogramBinCount } \, w
%\end{equation*}
%where \(w\) is the index of the leading edge to the
%\(\mnth{w}\) bin and \( 0 \le w \le  \HistogramBinCount \).

%\subsubsection{Histogram accumulation}

%\begin{equation*}
	%H[x,y] =  \Dim{ \BinFunc(\FeaturesName, x, y) }
%\end{equation*}
%where \(H[x,y]\) is a \(\HistogramBinCount \times \HistogramBinCount\) matrix and
%where \(\operatorname{Bin}(\FeaturesName, x, y)\) is the set of all tuples
%\((f_0, f_1) \in \FeaturesName\) such that
%\begin{enumerate}
	%\item \(b_0[x] \le f_0 < b_0[x+1]\) and
	%\item \(b_1[y] \le f_1 < b_1[y+1]\).
%\end{enumerate}



%\begin{algorithm}
	%\caption{Obtain vessellness features}\label{alg:seg_vessellness}
	%\begin{algorithmic}[1]
		%$\Call{Hessian}$
		%$\Call{Eigenvalues}$
		%\Comment{Frangi}
	%\end{algorithmic}
%\end{algorithm}

%\begin{algorithm}
	%\caption{Learn discriminant function}\label{alg:seg_discrim}
	%\begin{algorithmic}[1]
	%\end{algorithmic}
%\end{algorithm}

%\begin{algorithm}
	%\caption{Threshold}\label{alg:seg_threshold}
	%\begin{algorithmic}[1]
	%\end{algorithmic}
%\end{algorithm}

%\begin{algorithm}
	%\caption{Post-processing}\label{alg:seg_post}
	%\begin{algorithmic}[1]
	%\end{algorithmic}
%\end{algorithm}



%\subsection{Registration}

%% TODO FIXME
%This step is not a set part of the neuron tracing algorithm. This
%is meant to handle the subvolume.

%\subsection{Tracing}

% TODO
\begin{figure}
\centering
\resizebox{1.0\textwidth}{!}{\begin{tikzpicture}[node distance=0.5cm]
	\tikzstyle{step} = [draw,shape=circle,scale=0.4];
	%
	\node (step_shortest_path)        at (0,0)             [draw,step] {\parbox{3cm}{\centering{}Shortest path using fast-marching}};

	\node (step_centre_pts)          [right=of step_shortest_path]    [draw,step] {\parbox{2cm}{\centering{}Centre of dendrite}};
	\node (step_radii_estimation)    [right=of step_centre_pts]       [draw,step] {\parbox{2cm}{\centering{}Radii estimation}};
	\node (step_neumorph_repr)       [below=of step_radii_estimation] [draw,step] {\parbox{3cm}{\centering{}Neuron morphology representation}};
	\node (step_post)                [left=of step_neumorph_repr]    [draw,step] {\parbox{2cm}{\centering{}Post-processing of tracing}};

	\draw [->] (step_shortest_path)    -- (step_centre_pts);
	\draw [->] (step_centre_pts)       -- (step_radii_estimation);
	\draw [->] (step_radii_estimation) -- (step_neumorph_repr);
	\draw [->] (step_neumorph_repr)    -- (step_post);
%
\end{tikzpicture}
}
\caption[Diagram of tracing process]{\textbf{Diagram of tracing process}}\label{fig:proc-tracing}
\end{figure}



\section{Anticipating change}

Since the code for \gls{orionc} changes rapidly throughout the project, it is
necessary that the code structure is planned so that it does not have
to change too drastically as new components are added to the project. This
requires developing a project structure that does not require too many manual
changes which slow down development. In the following, various aspects of the
code structure and how they improve the code's ability to easily incorporate change.

\subsection{Directory structure}

The first part of setting up a new project is choosing a directory structure.
This determines where new files should go. A common structure is to create
separate directories for code for a library and code that will be compiled into
an executable. This is reflected in the directory structure where
%
\begin{description}[font=\computertextfamily\textpluscolon]
\item[lib] contains subdirectories which have all the source code that will be
	compiled together to create a library file (i.e., \computertext{liborion.a} on
	many Unix systems);
\item[lib/t] contains source files that have a matching directory layout to the
	rest of the \computertext{lib} directory so that it is easy to find the corresponding test to a given component
	(e.g., a library source file \computertext{lib/path/func.c} will have a
	corresponding test source file in \computertext{lib/t/path/func.c}); and
\item[src] contains source files that will be compiled into executables that can be run at the command-line.
\end{description}

\subsection{Build system}

% GNU make
Whenever a native build system is chosen, an important property is that it
should be portable. In order to fulfill this property, the build system is
written using GNU~\computertext{make} which is a portable version of the Unix
\computertext{make} build tool. This tool works by reading a
\computertext{Makefile} that lists the prerequisites for a given target file
and a set of commands needed to build those prerequisites. If the target is
older than any one of its prerequisites, then that target file is rebuilt. This
allows for testing changes to large projects without out needing to rebuild the
entire project.

% - automatic prerequisite scanning
Another property is that the build system should not have to require many
changes when adding new source code to the project.  This is accomplished by
using automatic dependency scanning. Before any code is compiled, the files are
scanned to create a list of prerequisites for each file. By scanning for
prerequisites, most of the files in the project do not need explicit rules in
the \computertext{Makefile} which means that when a new file is added to the
project, no changes to the build system are necessary.

An automatic build system allows for easily building the project on a new
machine. If the build system is portable and well-tested, the project should
build on the new system without manual intervention. When making a project for
scientific purposes, this is essential for reproducibility because the knowledge of
how to build the software is completely written down in an executable form.

\subsubsection{Configuration}

% - automatic dependency build flag configuration
Software projects do not live in isolation; many projects depend on outside
libraries to implement functionality. However, whenever a dependency is added,
this adds another point where the build could break. Outside libraries are not
necessarily in the same location on every system. In order to portably build the
\gls{orionc} with these outside libraries, the build system has automatic build
configuration that scans for the location of any dependencies that it needs.
This allows the same build system to be used on multiple systems without having
to use any fixed paths that are specific to a single computer.

% - automatic compiler extension configurator (for example, GCC branch prediction __builtin_expect() macro)
In addition, sometimes developers may want to enable system-specific features
in order to speed up their code. When this happens, the code must be able to
detect when such features are available and use a fallback if they are not
available. The build system for \gls{orionc} contains an automatic system
configuration scanner that it uses for this purpose. This configuration scanner
is currently only used to enable the GCC compiler's branch prediction macro
(\computertext{__builtin_expect}) which is used to give hints to the compiler
whether or not a given condition is likely or not~\autocite{drepper2007memory}.

\subsubsection{Debugging}

As the project progresses, the development is inevitably going to come across
bugs. When programming in C or C++, the class of bugs that occur are different
from the kinds that occur in MATLAB. These usually are related to invalid
memory access (e.g., buffer overflows, stack overflows). To help ease debugging,
the build system must also support tools that can help track these errors.
These tools often require the addition of build options that add extra metadata
to the compiled code. The build system thus supports
%
\begin{enumerate*}[label={}]
	\enumdescitem{code coverage builds} which are used to determine how
		much of the code is tested by the testing code;
	\enumdescitem{debug builds} to enable debugging symbols that can be
		used to stop and inspect the execution of a program and catch memory access violations; and
	\enumdescitem{profiling builds} which can be used to understand the
		areas of the code that are running slowly in order to intelligently decide whether to optimize it.
\end{enumerate*}
