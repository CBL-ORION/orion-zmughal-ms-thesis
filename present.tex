%% vim:tw=66:spell:wrap:ft=tex:
\ifx \printpresenthandout \undefined
	\ifx \printpresentarticle \undefined
		% no handout and no article
		\documentclass{beamer}
	\else
		% print article
		\documentclass[11pt]{article}
		\usepackage{beamerarticle}
	\fi
\else
	% handout
	\documentclass[handout]{beamer}
\fi
% Preamble
%\usepackage[small,sf,bf]{titlesec}
\usepackage[utf8]{inputenc}
\usepackage[british]{babel}

\usepackage[table,x11names,dvipsnames,rgb]{xcolor}
\usepackage{tikz}
\usetikzlibrary{snakes,arrows,shapes,automata}
\usetikzlibrary{positioning}

\usepackage{float}
\usepackage{graphicx}

%% use as
%%     \vertcenterimage{\includegraphics{*}}
\newcommand{\vertcenterimage}[1]{\raisebox{-.5\height}{#1}}

%% use as
%%     \flipbox{\includegraphics{*}}
\newcommand{\flipbox}[1]{\scalebox{1}[-1]{#1}}


\usetheme{Ilmenau}
\usecolortheme{beaver}
\usepackage{ifxetex}

\ifxetex
	% set font to Tahoma
	\usefonttheme{professionalfonts} % using non standard fonts for beamer
	\usefonttheme{serif} % default family is serif
	\usepackage{fontspec}
	\setmainfont{Tahoma}
\else
	\usepackage[T1]{fontenc}
\fi

\definecolor{CBLRed}{RGB}{205,0,0}

% Title on title slide
\setbeamerfont{title}{size = \Large}
\setbeamercolor{title}{fg = black, bg = white}

\setbeamercolor{frametitle}{fg=white,bg=CBLRed}
\setbeamercolor{section in head/foot}{fg=white,bg=CBLRed}
\setbeamercolor{subsection in head/foot}{fg=white,bg=CBLRed}
\setbeamercolor{author in head/foot}{fg=white,bg=CBLRed}
\setbeamertemplate{headline}{}
\beamertemplatenavigationsymbolsempty

\logo{}
\def\insertlogo
{%
	\color{gray}
	\begin{tabular*}{\paperwidth}{p{0.01\paperwidth}p{0.84\paperwidth}p{0.15\paperwidth}}
		&
		\parbox[c][][c]{\textwidth}{\raggedright
			\vertcenterimage{\includegraphics[width=\textwidth]{gfx/cbl-footer.png}}
		}
		&
		\parbox[c][][c]{\textwidth}{%\raggedleft
			{\large\insertframenumber} %/\inserttotalframenumber
		}
	\end{tabular*}
 }

%\usefonttheme[onlymath]{serif}
\setbeamertemplate{frametitle}[default][center]
\setbeamertemplate{footline}{}
%\setbeamertemplate{footline}[frame number]

%% Change beamer bullets to circles rather than the ball default
\setbeamertemplate{itemize items}[circle]
\setbeamertemplate{enumerate items}[circle]


\usepackage{textcomp}
\usepackage{fancyvrb}
\usepackage{changepage}
\usepackage{multicol}
\usepackage{wasysym}
\usepackage{listings}

\lstset{%basicstyle=\small\ttfamily,
%numbers=left,
%escapeinside=||
}
\newenvironment{indented}{\begin{adjustwidth}{1.5em}{}}{\end{adjustwidth}}

% http://tex.stackexchange.com/questions/12550/changing-default-width-of-blocks-in-beamer/12551#12551
\newenvironment<>{varblock}[2][.9\textwidth]{%
  \setlength{\textwidth}{#1}
  \begin{actionenv}#3%
    \def\insertblocktitle{#2}%
    \par%
    \usebeamertemplate{block begin}}
  {\par%
    \usebeamertemplate{block end}%
  \end{actionenv}}


%% TOC at beginning of sections/subsections
%\AtBeginSection[]
%{
%% <beamer>{Table of Contents}
%        \begin{frame}
%                \tableofcontents[
%                                currentsection,
%                                hideothersubsections,
%                                sectionstyle=show/shaded,
%                                subsectionstyle=hide/hide,
%                ]
%        \end{frame}
%}
%\AtBeginSubsection[]
%{
%   \begin{frame}
%                \begin{tabular}{p{0.3\paperwidth}p{0.7\paperwidth}}
%                        \tableofcontents[
%                                        currentsection,
%                                        hideothersubsections,
%                                        sectionstyle=show/shaded,
%                                        subsectionstyle=hide/hide,
%                        ]
%                &
%                        \begin{varblock}[0.5\paperwidth]{}
%                                \vspace{.2cm}
%                                \tableofcontents[
%                                        currentsubsection,
%                                        hideothersubsections,
%                                        sectionstyle=hide/hide,
%                                        subsectionstyle=show/shaded,
%                                ]
%                        \end{varblock}
%                \end{tabular}
%   \end{frame}
%}


\AtBeginSubsection[]
{
	\begin{frame}
		\begin{multicols}{2}
			\tableofcontents[
				currentsection,
				hideothersubsections,
				sectionstyle=show/shaded,
				subsectionstyle=show/shaded,
			]
		\end{multicols}
  \end{frame}
}

%% TikZ arrows
%% From <https://tex.stackexchange.com/questions/61507/drawing-arrows-in-beamer>
\tikzset{
    myarrow/.style={
        draw,
        fill=orange,
        single arrow,
        minimum height=3.5ex,
        single arrow head extend=1ex
    }
}
\newcommand{\arrowup}{%
\tikz [baseline=-0.5ex]{\node [myarrow,rotate=90] {};}
}
\newcommand{\arrowdown}{%
\tikz [baseline=-1ex]{\node [myarrow,rotate=-90] {};}
}
\newcommand{\arrowright}{%
\tikz [baseline=-0.5ex]{\node [myarrow,rotate=0] {};}
}
\newcommand{\arrowleft}{%
\tikz [baseline=-0.5ex]{\node [myarrow,rotate=180] {};}
}

\colorlet{tbackground}{blue!80}
\colorlet{tobjectives}{red!80}
\colorlet{tmetrics}{green!120}
\newcommand{\tbackground}[1]{\textcolor{tbackground}{#1}}
\newcommand{\tobjectives}[1]{\textcolor{tobjectives}{#1}}
\newcommand{\tmetrics}[1]{\textcolor{tmetrics}{#1}}
\begin{document}
% meta needs to be in \begin{document} so that tabular works
\title[Converting a Neuron Morphology Reconstruction System: Open Science Design and Implementation]
{Converting a Neuron Morphology Reconstruction System:\\Open Science Design and Implementation}
\author[Zakariyya Mughal]{%
\begin{tabular}{r@{ }l}%
Author:    & Zakariyya Mughal \\[1ex]
Committee: & Dr. Ioannis A. Kakadiaris \\
           & Dr. Emanuel Papadakis \\
           & Dr. Shishir Shah
\end{tabular}%
}
\institute[cbl.uh.edu]{Computational Biomedicine Lab\\University of Houston}
\date{2016 January 15}

%%  - List of committee members (start with advisor and arrange others in alphabetical order of last name).
%\begin{frame}
%\frametitle{Committee}
%\begin{itemize}
%        \item Dr. Ioannis A. Kakadiaris; Dept. of Computer Science
%        \item Dr. Emanuel Papadakis; Dept. of Mathematics
%        \item Dr. Shishir Shah; Dept. of Computer Science
%\end{itemize}
%\end{frame}

\setbeamercovered{transparent}
%  - Title slide.%{{{
\frame{\titlepage}
%}}}

The following text is color-coded based on what kind of
information it provides the audience:
\tbackground{background information},
\tobjectives{objectives},
\tmetrics{metrics for completion of objectives}.

\section{Goal}
%  - Motivation - short introduction to your research, pointers to applications, why the research is important.
\begin{frame}\frametitle{\secname}
	\centering
	\tobjectives{
	The reimplementation of  a system
	%
		for \alert{neuron reconstruction}
	%
		that is distributed as an \alert{open-science} project
		and
	%
		compatible with the \alert{BigNeuron} project.
	%
	}
	\end{itemize*}
\end{frame}

\subsection{Neuron reconstruction}
\begin{frame}\frametitle{\subsecname}
	% TODO
	[\tbackground
		{
			what is neuron reconstruction; data acquisition; image analysis; why
			automate (i.e., why automated neuron tracing research is important)
		}
	]
\end{frame}

\subsection{Open science}
\begin{frame}\frametitle{\subsecname}
	% TODO
	[\tbackground
		{
		what it means to be an open-science project (i.e., open release of all
		research artifacts as early as possible)
		}
	]
\end{frame}

\subsection{BigNeuron}
\begin{frame}\frametitle{\subsecname}
	% TODO
	[\tbackground
		{
			goal of BigNeuron and purpose of participation in
			BigNeuron
		}
	]
\end{frame}

\section{Problem}
%  - Problem Statement - Goals, objectives, aims. May also include challenges.
%        IMPORTANT: State your specific aims clearly - what you plan to achieve and how
%        you plan to achieve it. Let your contributions to the field of Computer Science
%        be clear.
\begin{frame}\frametitle{\secname}
	[\tbackground{explain how the ORION code exists in MATLAB form but is difficult to
	integrate into BigNeuron}; explain challenges in later slides]
\end{frame}

\subsection{Objectives}
\begin{frame}\frametitle{\subsecname}
	% TODO
	\tobjectives{
	Taken from abstract. TODO: make concise
	\begin{enumerate}
		\item analyze the ORION algorithm and implementation to
			determine the architecture for the new system that is
			efficient and extensible;
		\item integrate the system into a popular toolkit for biomedical
			image analysis for ease-of-use and visualization;
		\item develop a test suite of both the individual components (unit
			testing) and across the whole system (integration tests);
			and % "and" before last item
		\item ensure that the software gives reproducible results by
			making it easy to build and distribute. % end of sentence
	\end{enumerate}
	}
\end{frame}

\section{Related work}
%  - Related Work - a catchy name is “State of the Art”.
%        Literature review relevant to specific research. Your committee
%        members are expected to already have a general background in your field.
%        Concentrate on the details of your specific research area. It is preferable to
%        use more recent literature (except for some classical “old” ones). Also avoid
%        reviewing standard concepts that have been around for a long time.
\begin{frame}\frametitle{\secname}
	\tbackground{
	\begin{itemize}
		\item Prior work in \alert{neuron reconstruction}
		\item Principles of \alert{open-science}
		\item An overview of \alert{BigNeuron}
	\end{itemize}
	}
\end{frame}

\subsection{Neuron reconstruction}
\begin{frame}\frametitle{\subsecname}
	% TODO
	[\tbackground{explain the problem of neuron reconstruction; history of
	neuron tracing; biological relevance and applications}]
\end{frame}

\subsubsection{DIADEM Challenge}
\begin{frame}\frametitle{\subsubsecname}
	% TODO
	[\tbackground{explain DIADEM Challenge; data sets; outcomes}]
\end{frame}

\subsubsection{Metrics}
\begin{frame}\frametitle{\subsubsecname}
	% TODO
	[\tbackground{Create a table of metrics (e.g., precision, recall,
	DIADEM metric, btmorph, NetMets) and explain how they work}]
\end{frame}

\subsubsection{Methods}
\begin{frame}\frametitle{\subsubsecname}
	% TODO
	[\tbackground{Create a table of methods: what data they work with (2D/3D), modality,
	metrics, what datasets they validated against}]
\end{frame}

\subsection{Open science}
\begin{frame}\frametitle{\subsecname}
	% TODO
	[\tbackground{explain open science}]
\end{frame}

\subsubsection{``Bermuda Principles'''}
\begin{frame}\frametitle{\subsubsecname}
	% TODO
	[\tbackground{explain ``Bermuda Principles'' from Human Genome Project
	(early large-scale open research project)}]
\end{frame}

\subsubsection{Open science projects}
\begin{frame}\frametitle{\subsubsecname}
	% TODO
	[\tbackground{showcase a few open science projects to show breadth}]
\end{frame}

\subsection{BigNeuron}
\begin{frame}\frametitle{\subsecname}
	% TODO
	[\tbackground{explain what BigNeuron is; how it relates to this thesis;
		neuron stitching; bench testing}]
\end{frame}

\subsubsection{Vaa3D}
\begin{frame}\frametitle{\subsecname}
	% TODO
	[\tbackground{explain how Vaa3D has a plugin architecture and how it
	standardizes the BigNeuron project}]
\end{frame}

\section{Proposed framework}
%  - Proposed Framework - this is the most important part of your proposal.
%        Expect most questions from committee members on this. Put in all your
%        best here. Describe the entire framework before separating into work-done and
%        work-left. Do not mix framework with results. You may start with a one-slide
%        overview of the entire framework before going into details. Make sure committee
%        members will be able to understand your work. Relate to previous approaches and
%        how your is an extension or a difference. Include mathematical equations that
%        will explain your work.
%  - Explain the relevance of each equation. Understand the meaning and
%        significance of each term in the equation irrespective of whether you define
%        them in the slide or not. TIP: Rather than doing your equation twice (for your
%        PPT and Latex write-up), you may do it once in your Latex document and copy
%        into TexPoint for the PPT.
\begin{frame}\frametitle{\secname}
	[\tobjectives{reiterate the objectives}]
\end{frame}

\subsection{Conversion}
\begin{frame}\frametitle{\subsecname}
	[\tbackground{explain difficulties with a rewrite vs. a refactor}]
\end{frame}

\subsubsection{Conversion challenges}
\begin{frame}\frametitle{\subsubsecname}
	[\tbackground{challenges when doing a rewrite from MATLAB to native code}]
\end{frame}

\subsubsection{Call graph}
\begin{frame}\frametitle{\subsubsecname}
	[\tbackground{analysis of existing codebase; how this informs the new
	implementation; why use the call graph instead of
	attempting to either do a clean room implementation (i.e.,
	by only reading papers) or creating a new architecture
	from scratch}]
\end{frame}

\subsection{Integration}
\begin{frame}\frametitle{\subsecname}
	[\tbackground{how to integrate with Vaa3D}\tmetrics{a
		demonstration video; visual results and metrics from using
	Vaa3D with neuron data from DIADEM; demonstration of any further analysis
	that can be done in Vaa3D}]
\end{frame}

\subsection{Testing and Reproducibility}
\begin{frame}\frametitle{\subsecname}
	[\tmetrics{dependency tracking by keeping version controlled copies
		of dependencies; documentation generation; continuous
	integration and autmotic builds, code coverage}]
\end{frame}

\subsubsection{Dependency tracking}
\begin{frame}\frametitle{\subsubsecname}
	[screenshots of all repos of deps used under the CBL-ORION
	namespace on GitHub
	\begin{itemize}
		\item ITK
		\item Vaa3D
		\item kiss-fft
	\end{itemize}
	]
\end{frame}


\section{Results}
%  - Preliminary Results - if you already have some results, please show it.
%        Talk about how you propose to validate your results. Show how your
%        results compare with previous results OR how it performs in general, (e.g., in
%        how many images it succeeded or in how many it failed). A chart will be useful.
%        Any statistical comparison should indicate level of significance.
%  - Present status of the proposed framework
%        - which aspects are completed, which are in progress, and which have not yet been done at all.
%  - It is advisable to include a timeline of future indicating milestones and when you hope to achieve each.
%  - Expected Impact - mention specific application areas where you have tried or will try your proposed framework.
%        Also, explain how it can be generalized to other areas in Computer
%        Science. What is the overall benefit for the community? What is the
%        contribution to Computer Science research?
\begin{frame}\frametitle{\secname}
	% TODO
\end{frame}

\subsection{Directory structure}
\begin{frame}\frametitle{\subsecname}
	% TODO
\end{frame}

\subsection{Tracing-based comparison}
\begin{frame}\frametitle{\subsecname}
	% TODO
\end{frame}

\subsection{Continuous integration}
\begin{frame}\frametitle{\subsecname}
	% TODO
\end{frame}

\subsection{Dependency tracking}
\begin{frame}\frametitle{\subsecname}
	% TODO
\end{frame}

\section{Conclusion}
%  - Conclusion slide could be catchy, something they will remember.
%        Maybe an animation, video, or terse statement that summarizes it all.


%  - Publications - you should list your publications or mention them in relevant sections of the presentation.


\end{document}
