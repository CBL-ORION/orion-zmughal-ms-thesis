%% vim:tw=66:spell:wrap:ft=tex:fdm=marker
% Preamble {{{
\documentclass[%
        hyperref={%
                pdfauthor={Zakariyya Mughal},%
                pdfpagemode={None},pdfpagelayout={SinglePage}}%
        xcolor={x11names},%
]{beamer}
%\usetheme{Warsaw}
\usecolortheme{beaver}
\usepackage{textcomp}
\usepackage{fancyvrb}
\usepackage{changepage}
\usepackage{multicol}
\usepackage{wasysym}
\usepackage[T1]{fontenc}
\usepackage{listings}

\lstset{%basicstyle=\small\ttfamily,
%numbers=left,
%escapeinside=||
}
\newenvironment{indented}{\begin{adjustwidth}{1.5em}{}}{\end{adjustwidth}}

\usepackage{tikz}
\usetikzlibrary{snakes,arrows,shapes,automata}

\title[\ensuremath{\mathrm{ORION}^c}]{Converting a Neuron Morphology Reconstruction System: Open Science Design and Implementation}
\author{Zakariyya Mughal}
\institute{Computational Biomedicine Lab\\University of Houston}
\date{2015 Nov 30}
%}}}
\begin{document}

%  - Title slide.
\frame{\titlepage}

%  - List of committee members (start with advisor and arrange others in alphabetical order of last name).
\section{Committee}
\begin{frame}
\end{frame}

%  - Motivation - short introduction to your research, pointers to applications, why the research is important.
\section{Motivation}
\begin{frame}
\end{frame}

%  - Problem Statement - Goals, objectives, aims. May also include challenges. IMPORTANT: State your specific aims clearly - what you plan to achieve and how you plan to achieve it. Let your contributions to the field of Computer Science be clear.
\section{Problem statment}
\begin{frame}
\end{frame}

%  - Related Work - a catchy name is “State of the Art”. Literature review relevant to specific research. Your committee members are expected to already have a general background in your field. Concentrate on the details of your specific research area. It is preferable to use more recent literature (except for some classical “old” ones). Also avoid reviewing standard concepts that have been around for a long time.
\section{Related work}
\begin{frame}
\end{frame}

%  - Proposed Framework - this is the most important part of your proposal. Expect most questions from committee members on this. Put in all your best here. Describe the entire framework before separating into work-done and work-left. Do not mix framework with results. You may start with a one-slide overview of the entire framework before going into details. Make sure committee members will be able to understand your work. Relate to previous approaches and how your is an extension or a difference. Include mathematical equations that will explain your work.
%  - Explain the relevance of each equation. Understand the meaning and significance of each term in the equation irrespective of whether you define them in the slide or not. TIP: Rather than doing your equation twice (for your PPT and Latex write-up), you may do it once in your Latex document and copy into TexPoint for the PPT.
%  - Preliminary Results - if you already have some results, please show it. Talk about how you propose to validate your results. Show how your results compare with previous results OR how it performs in general, (e.g., in how many images it succeeded or in how many it failed). A chart will be useful. Any statistical comparison should indicate level of significance.
%  - Present status of the proposed framework - which aspects are completed, which are in progress, and which have not yet been done at all.
%  - It is advisable to include a timeline of future indicating milestones and when you hope to achieve each.
%  - Expected Impact - mention specific application areas where you have tried or will try your proposed framework. Also, explain how it can be generalized to other areas in Computer Science. What is the overall benefit for the community? What is the contribution to Computer Science research?
%  - Conclusion slide could be catchy, something they will remember. Maybe an animation, video, or terse statement that summarizes it all.
%  - Publications - you should list your publications or mention them in relevant sections of the presentation. 

\end{document}
