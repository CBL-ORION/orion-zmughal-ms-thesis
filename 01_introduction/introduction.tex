\begin{savequote}[0.55\linewidth]
	\begin{fancyquote}
		Each discovery made by an investigator in a basic research
		laboratory has much larger implications today. The sum of the work in basic
		biology represents a rapidly expanding tool kit for engineers and inventors to
		use to construct items of value to society.
	\end{fancyquote}
	\qauthor{David Baltimore in \emph{How Biology Became an Information Science}, 2001}
	% TODO cite from \emph{The Invisible Future: The Seamless Integration of Technology Into Everyday Life}
\end{savequote}
\chapter{Introduction}\label{ch:introduction}


This thesis aims to detail the design and implementation of a
neuron morphology extraction system. This system is the result of
converting existing ORION 3\cite{ORION_Santamaria-Pang2015} from
MATLAB\cite{MATLAB:2013a} code to native C/C++ code.
This conversion process requires an analysis of the existing code
to determine. In addition, a rewrite allows for more thorough
design and testing which is necessary to verify that the code
behaves as expected and will continue to do so in the future. A
more complete overview of the goals and motivation of this work
are given in the following section.

\section{Motivation}
%  [ Discussion of the application area and context of the
%    project. ]

% TODO expansion of the ideas in the abstract

\subsection{Scientific software}
{ % how software has become important in experimental science
	Quantitative methods are an essential part of scientific research. The
	experimental sciences depend on the dissemination of the methods used in a
	study so that it can be clear how results are obtained and analysed. With the
	rapid increase in computing power, storage, and availability, it has become
	easier to collect and process larger and more complex datasets using
	sophisticated methods. Some common approaches to this change is to publish either a
	description of the algorithm or refer to software that can be obtained separately
	(either in binary or source code form).
}

Despite these approaches, there are still several challenges to a successfully
reproducing a published method.
{ % problems with a textual description
	A textual description of an algorithm is rarely a complete description of
	how an algorithm is implemented. Sophisticated methods often have tiny details
	that are mistakenly be left out that may be essential for the rest of the
	processing. One area that this occurs is in a data preprocessing and annotation
	stage where human interaction and assumptions are not written down.
}
{ % problems with software distribution
	Referencing readily available software packages gives
	other researchers direct access to the original method
	used. However, even here, there can be problems. Firstly,
	even before running the software, it is important to
	ensure that the
	software is available years later. This means not only the software itself, but
	all its dependencies. This can quickly become complicated as technology
	advances: changes in the Application Programming Interface (API) or Application
	Binary Interface (ABI) can cause both software distributed in source form or
	binary form. Both source code and binary software can have
	dependencies on platforms (e.g., operating systems,
	runtimes, computer architecture, etc.) and licensing of
	components that can hinder others from using the software.
}

{ % backwards compatibility and digital archival
	Even more precarious is when a dependency used by the
	software no longer behaves the same way as it did in
	previous versions. This can lead to software that appears
	to run, but gives unintended results.  Maintaining
	backwards compatibility for software is difficult since
	there are many parts to a non-trivial software system.
	% Dependency hell
	Resolving the exact versions of libraries and toolchains
	needed to build and run can be frustrating and is commonly
	known as \emph{dependency hell}. This problem can become
	daunting when dealing with multiple platforms and many
	libraries. As newer versions of these libraries are
	released, the maintenance stage of the \emph{systems
	development life cycle} becomes more important.
	Unmaintained software is prone to what is known as
	\emph{bitrot} --- that is, the process through which
	software that was working no longer works due to changes
	in the surrounding software ecosystem.
	There has been some work to prevent bitrot by recording a
	static copy of the software environment, but the digital
	archival (comprising both computing machinery and
	software) is in its infancy and has not caught up to that
	of paper-based materials.
}


\subsection{Open science and scientific software engineering}

As science becomes more oriented towards using computational
tools, the ideals of reproducibility and statistical hypothesis
testing become more difficult to achieve. In recent years, there
has been a push by researchers to incorporate \emph{open science}
practices in their work. One prominent advocate for open science
defines it as
\begin{quote}
	\begin{fancyquote}
	Open science is the idea that scientific knowledge of all kinds
	should be openly shared as early as is practical in the discovery
	process.
	\end{fancyquote}
	\qauthor{Michael Nielsen\footnote{http://michaelnielsen.org/blog/open-science-2/}} % TODO add more info about quote
\end{quote}

% Principles of open science <http://openscienceasap.org/open-science/>
%
%   Open Methodology
%   Open Source
%   Open Data
%   Open Access
%   Open Peer Review
%   Open Educational Resources

% <https://github.com/openscienceASAP/open-science-sticker>


% TODO
Software Carpentry...


\subsection{Open neuroscience}
% TODO
Biology and neuroscience in particular have been using more
quantitative imaging tools: patch clamp, EEG, fMRI, fluorescent
microscopy ... need to measure what is being looked at ... perform
statistical analysis across samples

% TODO
Allen Institute for Brain Science,
Big Brain project,
Big Neuron
NITRC
Neuinfo  Framework
Neuromorpho
\citep{FrontNeuroinform:NeuroDebian}

\begin{enumerate}[(i)]
\item to evaluate the MATLAB codebase of ORION3 and determine how to structure the new codebase
\item integrate the
\end{enumerate}

\section{Literature review}

% TODO
%
% - refactoring and conversion of code
% - Open science
% - bioinformatics

\section{Survey of libraries used}

% - cite the architecture of open-source applications
% - ITK
% - VTK

