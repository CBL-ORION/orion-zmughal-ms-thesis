% TODO Analysis
% Project objectives:
% - Integrate with Vaa3D tool for biomedical image analysis:
%   - create a plugin
% - Convert the MATLAB code to C/C++ native code:
%   - easier to distribute with Vaa3D
% - Release this code as free/open source software
%   - for use with BigNeuron open science project
% - Build a test suite
%   - testing multiple software environments for portability and reproducibility
% 
% Challenges and risks:
%
% - Starting without a test suite
%   - needs to be created using properties of the original code
% - Memory management
%   - automatic versus manual management
% - Data layout differences of n-dimensional arrays
%   - MATLAB's column-major 1-indexed versus C's row-major 0-indexed
% - Data handling differences:
%   - original code makes use of caching and subvolumes for saving time and memory respectively

% TODO Design
% - use the initial MATLAB code as a design and refactor
%   ( because changing too many things at once makes organisation
%   difficult )
% - automated GNU make-based build system:
%   provides code coverage, debug builds, profiling builds,
%   AddressSanitizer
% - automatic compiler extension configurator (for example, GCC branch prediction __builtin_expect() macro)
% - automatic dependency build flag configuration
% - automatic prerequisite scanning

% TODO Implementation
%
% seg -> registration -> tracing
%  |         |              |--------------------------> ... -----------> ...
%  |         |
%  |         |--------------------> .... -------> ....
%  v
% apply filter at multiscale
%  |
%  |
%  |
%  v
% get scale for max response
%  |
%  |
%  |
%  v
% get vessellness features at that scale
%  |
%  |
%  |
%  v
% discriminant
%
% - ITK library (interesting because mixing C with C++)
% - VTK and Vaa3D for visualisation
% - storage : pixel_type (float32) by default to save space
% - internal calculations: float64
% - all calculations are done in memory (could use mmap in the future)
% - ndarray
% 
%
% usage of GitHub pull requst
% each subcomponent is implemented on a branch
%  | master  |
%       |
%       |---------------> | seg |
%                            |---------------> | seg_laplacian |
%                                                     |---------------> | seg_fftn |
%                               


% TODO testing
% - use the Test Anything Protocol (TAP) : allows for standardised processing
% - every time comits are pushed, they areted under gcc and clang
% - A GitHub pull request and is used to track the progress on the subomponent via Travis-CI
%   until it passes
% - isolates breaking changes to that branch
% - devops: automatic provision of build environment: no manual build instructions
% - tracing-based comparsiion to original MATLAB output
%   - implements MATLAB algorithm as-in when using compile time flag
%   - otherwise a version that may contain bug fixes is compiled
% - assertions for debug-build throughout code so that programmer
%   pre-conditions and post-conditions are specified
% - all external dependencies are managed and versioned in auxiliary
%   repositories so that the exact same build can be built in case upstream
%   changes occur

% TODO conclusion
% - software is a changing system
% - rewrites across languages are difficult especially if they differ as much as MATLAB and C do
% -best to approach with rewrite close to original folowed by refactoring
% - write automated tools
%   - less work for deployment
%   - fewer instructions and assumptions are written down

